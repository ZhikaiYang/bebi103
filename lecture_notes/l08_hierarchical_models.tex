In this lecture, we will investigate \textbf{hierarchical models}, in
which some model parameters are dependent on others in specific ways.
This is best learned by example.

% %%%%%%%%%%%%%%%%%%
\subsection{A hierarchical model example}
In
\href{http://bebi103.caltech.edu/2015/tutorials/t3b_boolean_data.html}{Tutorial
  3b}, we studied reversals under exposure to blue light in
\textit{C. elegans} with Channelrhodopsin in two different neurons.
Let's consider one of the strains which contains a Channelrhodopsin in
the ASH sensory neuron.  We found that 9 out of 35 worms reversed
under exposure to blue light.  We used this measurement to estimate
the probability $p$ of reversal.  Specifically, we found that the
posterior probability of reversal given $r$ our of $n$ trials showed
reversals was\footnote{In Tutorial 3b, we used $n_r$ for the number of
  reversals.  We use $r$ here because we will have some more
  subscripts and we want to keep notation clean.}
\begin{align}
P(p\mid r, n, I) = \left\{\begin{array}{ccl}
\displaystyle{\frac{(n+1)!}{(n-r)!r!}\,p^{r}(1-p)^{n-r}} && 0\le p \le 1 \\[1em]
0 & & \text{otherwise}.
\end{array}
\right.
\label{eq:boolean_posterior}
\end{align}
This posterior assumed a uniform prior $P(p\mid I)$ on $0\le p \le 1$,
and a binomial likelihood, $P(r\mid n, p , I)$.

Next year, we will do the experiment again.  Actually, we could image
doing the experiment over and over again, each time getting a value of
$r$ and $n$.  Conditions may change from experiment to experiment.
For example, we may have different microscope set-ups, slight
differences in the strain of worms we're using, etc.  We are left with
some choices on how to model the data.


% %%%%%%%%%%%%%%%%%%
\subsubsection{Pooled data: identical parameters}
We could pool all of the data together.  In other words, let's say we
measure $r_1$ out of $n_1$ reversals in the first set of experiments,
$r_2$ out of $n_2$ reversals in the second set, etc., up to $k$ total
experiments.  We could pool all of the data together to get
\begin{align}
r = \sum_{i=1}^k r_i \text{ out of } n = \sum_{i=1}^k n_i \text{ reversals}.
\end{align}
We then compute our posterior as in equation
\eqref{eq:boolean_posterior}.  Here, the assumption is that the result
in each experiment are governed by \textit{identical parameters}.
That is to say that we assume $p_1 = p_2 = \cdots = p_k = p$.

This is similar to what we did in section \ref{sec:l01_learning}, in
which we looked at how a single hypothesis (or parameter value) is
informed by more data.


% %%%%%%%%%%%%%%%%%%
\subsubsection{Independent parameters}
As an alternative, we could instead say that the parameters in each
experiment are totally independent of each other.  In this case, we
assume that $p_1$, $p_2$, $\ldots$, $p_k$ are all independent of each
other.  Thus, the posterior probability is
\begin{align}
P(\mathbf{p}\mid \mathbf{r},\mathbf{n}, I) = \prod_{i=1}^k
\frac{(n_i+1)!}{(n_i-r_i)!r_i!}\,p_i^{r_i}(1-p_i)^{n_i-r_i},
\end{align}
where $\mathbf{p} = \{p_1, p_2, \ldots p_k\}$, with $\mathbf{n}$ and
$\mathbf{r}$ similarly defined, and the posterior is understood to be
zero if any the $p_i$'s fall our of the interval $[0,1]$.  

When we make this assumption, we often report a value of $p$ that is
given by the mean of the $p_i$'s with some error bar.


% %%%%%%%%%%%%%%%%%%
\subsubsection{Best of both worlds: a hierarchical model}
Each of these extremes have their advantages.  We are often trying to
estimate a parameter that is more universal that out experiments,
e.g., something that describes worms with Channelrhodopsin in the ASH
neuron generally.  So, pooling the experiments makes sense.  On the
other hand, we have reason to assume that there is going to be a
different value of $p$ in different experiments, as biological systems
are highly variable, not to mention measurement variations.  So, how
can we capture both of these effects.

We can consider a model in which there is a ``master'' reversal
probability, which we'll call $q$ to avoid too many $p$'s, and the
values of $p_i$ may vary from this $p$ according to some probability
distribution, $P(p_i\mid q, I)$.  So now, we have parameters
$p_1, p_2, \ldots, p_k$ and $q$.  So, the posterior can be written
using Bayes's theorem,
\begin{align}
P(q,\mathbf{p}\mid \mathbf{r}, \mathbf{n}, I) 
= \frac{P(\mathbf{r},\mathbf{n}\mid q, \mathbf{p}, I)\,
P(q, \mathbf{p}\mid I)}{P(\mathbf{n}, \mathbf{r}\mid I)}.
\end{align}
Note, though, that the observed values of $r$ do not depend directly
on $q$, only on $\mathbf{p}$.  In other words, they only depend
indirectly on $q$.  So, we can write
$P(\mathbf{r},\mathbf{n}\mid q, \mathbf{p}, I) =
P(\mathbf{r},\mathbf{n}\mid \mathbf{p}, I)$.  Thus, we have
\begin{align}
P(q,\mathbf{p}\mid \mathbf{r}, \mathbf{n}, I) 
= \frac{P(\mathbf{r},\mathbf{n}\mid \mathbf{p}, I)\,
P(q, \mathbf{p}\mid I)}{P(\mathbf{n}, \mathbf{r}\mid I)}.
\end{align}
Next, we can rewrite the prior using the definition of conditional
probability.
\begin{align}
P(q,\mathbf{p}\mid I) = P(\mathbf{p}\mid q,I)\, P(q\mid I).
\end{align}
Substituting this back into our expression for the posterior, we have
\begin{align}
P(q,\mathbf{p}\mid \mathbf{r}, \mathbf{n}, I) 
= \frac{P(\mathbf{r},\mathbf{n}\mid \mathbf{p}, I)\,
P(\mathbf{p}\mid q,I)\, P(q\mid I)}{P(\mathbf{n}, \mathbf{r}\mid I)}.
\end{align}
Now, if we read off the numerator of this equation, we see a chain of
dependencies.  The experimental results $\mathbf{r}$ depend on
parameters $\mathbf{p}$.  Parameters $\mathbf{p}$ depend on
\textit{hyperparameter} $q$.  Hyperparameter $q$ then has some prior
distribution.  Any model that can be written as a chain of
dependencies like this is called a \textbf{hierarchical model}, and
the parameters that do not directly influence the data are called
\textbf{hyperparameters}.

So, the hierarchical model captures both the experiment-to-experiment
variability, as well as the master regulator of outcomes.  Note that
the product $P(\mathbf{p}\mid q,I)\, P(q\mid I)$ comprises the prior,
and it is therefore independent of the data.


% %%%%%%%%%%%%%%%%%%
\subsection{Exchangeability}
The conditional probability, $P(\mathbf{p}\mid q, I)$, can take any
reasonable form.  In the case where we have no reason to believe that
we can distinguish any one $p_i$ from another prior to the experiment,
then the label ``$i$'' applied to the experiment may be exchanged with
the label of any other experiment.  I.e.,
$P(p_1, p_2, \ldots, p_k \mid q, I)$ is invariant to permutations of
the indices.  Parameters behaving this way are said to be
\textbf{exchangeable}.  A common (simple) exchangeable distribution is
\begin{align}
P(\mathbf{p}\mid q, I) = \prod_{i=1}^k P(p_i\mid q, I),
\end{align}
which means that each of the parameters is an independent sample out
of a distribution $P(p_i\mid q)$, which we often take to be the same
for all $i$.  This is reasonable to do in the worm reversal example.


% %%%%%%%%%%%%%%%%%%
\subsection{Choice of the conditional distribution/prior}
We need to specify out prior, which for this hierarchical model means
that we have to specify the conditional distribution,
$P(p_i\mid q, I)$, as well as $P(q\mid I)$ For the latter, we will
take it to be uniform on $[0, 1]$.  For the conditional distribution,
we will assume it is beta-distributed, which is defined on the
interval $[0,1]$ and can be peaked.  The beta distribution can be
written as
\begin{align}
P(p\mid \alpha, \beta) = \frac{\Gamma(\alpha+\beta)}{\Gamma(\alpha)\Gamma(\beta)}\,
p^{\alpha-1}(1-p)^{\beta-1},
\end{align}
where it is parametrized by positive constants $\alpha$ and $\beta$.
If $\alpha$ and $\beta$ are both greater than unity, the distribution
is peaked, and the mode is
\begin{align}
  p^* \equiv \omega = \frac{\alpha - 1}{\alpha + \beta - 2}.
\end{align}
The ``concentration,'' $\kappa = \alpha + \beta$, of the distribution
describes its spread.  As $\kappa$ gets larger, the distribution
becomes tighter.  So, we might want to think of the conditional
distribution in terms of $\omega$ and $\kappa$.  We can convert back
to $\alpha$ and $\beta$ using
\begin{align}
  \alpha &= \omega(\kappa - 2) + 1 \\
  \beta &= (1-\omega)(\kappa - 2) + 1.
\end{align}
We have $0 < \omega < 1$ and $2 < \kappa$.  A reasonable model would
be to take $\omega = q$ with some concentration $\kappa$.  This gives
an additional hyperparameter, $\kappa$, which describes
experiment-to-experiment variability.  We will take
$P(\kappa\mid I) \propto 1/\kappa$, as we typically do for scale
parameters.  Thus, our full posterior is
\begin{align}
P(q, \kappa,\mathbf{p}\mid \mathbf{r}, \mathbf{n},I) \propto
P(\mathbf{r},\mathbf{n}\mid \mathbf{p}, I)\,
\kappa^{-1}\,\left(\prod_{i=1}^k P(p_i\mid q,\kappa)\right),
\end{align}
nonzero on $0\le q,\mathbf{p} \le 1$ and $\kappa > 2$, where
\begin{align}
P(p_i\mid q, \kappa) =  \frac{\Gamma(\kappa)}{\Gamma(q(\kappa - 2) + 1)\Gamma((1-q)(\kappa - 2) + 1)}\,
p^{q(\kappa - 2)}(1-p)^{(1-q)(\kappa - 2)}.
\end{align}
As before, we have a binomial likelihood, where we assume the
experiments are independent.
\begin{align}
P(\mathbf{r},\mathbf{n}\mid \mathbf{p}, I) = \prod_{i=1}^k \frac{n_i!}{(n_i-r_i)!r_i!}\,
p_i^{r_i}(1-p_i)^{n_i-r_i}.
\end{align}


% %%%%%%%%%%%%%%%%%%
\subsection{Implementation}
In some cases, we can do some macho integration and work out
analytical results for the posterior of a hierarchical model.  This
usually involves choosing conjugate priors.  Most often, though, we
need to resort to numerical methods.  To see the worm reversal problem
solved with a hierarchical model, see the implementation
\href{http://bebi103.caltech.edu/2015/tutorials/l08_hierarchical_models.html}{here}.


% %%%%%%%%%%%%%%%%%%
\subsection{Generalization}
The worm reversal problem is easily generalized.  You can imagine
having more levels of the hierarchy.  This is just more steps in the
chain of dependencies that are factored in the prior.  For general
parameters $\boldsymbol{\theta}$ and hyperparameters
$\boldsymbol{\phi}$, we have
\begin{align}
P(\boldsymbol{\theta}, \boldsymbol{\phi} \mid D, I) = \frac{P(D\mid \boldsymbol{\theta}, I)\, P(\boldsymbol{\theta} \mid \boldsymbol{\phi}, I)\,P(\boldsymbol{\phi}\mid I)}
{P(D\mid I)}
\end{align}
for a two-level hierarchical model.  As we have seen in the course,
the work is all in coming up with the models for the likelihood
$P(D\mid \boldsymbol{\theta}, I)$ and prior,
$P(\boldsymbol{\theta} \mid \boldsymbol{\phi},
I)\,P(\boldsymbol{\phi}\mid I)$.
For coming up with the conditional portion of the prior,
$P(\boldsymbol{\theta} \mid \boldsymbol{\phi}, I)$, we often assume a
Gaussian distribution because this often describes
experiment-to-experiment variability.  Bayes's theorem gives you the
posterior, and it is then ``just'' a matter of computing it of
sampling from it.
