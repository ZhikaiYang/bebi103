We have spent a lot of time in the past couple of weeks looking at the
problem of parameter estimation.  Really, we have been stepping
through the process of bringing our thinking about a biological system
into a concrete model (of type 3, in our language) that defines a
likelihood for the data and the parametrization thereof.  Writing down
Bayes's theorem then gives the posterior,
\begin{align}
P(\mathbf{a}\mid D, I) = 
\frac{P(D\mid \mathbf{a},I)\,P(\mathbf{a}\mid I)}{P(D\mid I)},
\end{align}
where $\mathbf{a}$ is the set of parameters.  Solving the parameter
estimation problem involves computing the posterior, which usually
involves summarizing the posterior into a form that can be processed
intuitively.

% %%%%%%%%%%%%%
\subsection{Adding models to the probabilities}
When we write Bayes's theorem for the parameter estimation problem,
implicit in the definition of the likelihood is the fact that we are
using a specific model. (Again, we're talking about model level 3
here, which is what we need to specify our data analysis problem.)  We
really should be explicit and include which model we're using in our
probabilities.  Let $M_i$ be model $i$.  Then, for parameter
estimation, we have
\begin{align}
P(\mathbf{a}_i\mid D, M_i, I) = 
\frac{P(D\mid \mathbf{a}_i, M_i, I)\,P(\mathbf{a}_i\mid M_i, I)}{P(D\mid M_i, I)}.
\label{eq:param_est}
\end{align}
Notice that we have also assigned the subscript $i$ to the set of
parameters we are determining to specify that they are associated with
model $M_i$.  So this is a more explicit description of the
probabilities associated with the parameter estimation problem.


% %%%%%%%%%%%%%
\subsection{Probabilities of models}
Remember that Bayesian probability is a measure of the plausibility of
any logical conjecture.  So, we can talk about the probability of
models being true.  So, what is the probability that a model is true,
given the observed data?  Again, this is given by Bayes's theorem.
\begin{align}
P(M_i\mid D, I) = \frac{P(D\mid M_i, I)\,P(M_i\mid I)}{P(D\mid I)}.
\end{align}
This is Bayes's theorem states for the model selection problem.  Let's
look at each term in turn.
\begin{itemize}
\item $P(M_i\mid D, I)$, as we said before, is the probability that
  model $M_i$ is true given the measured data.
\item $P(D\mid I)$ is a normalization constant for the posterior that
  is computed by marginalizing over all possible models
  \begin{align}
    \sum_i P(M_i\mid D, I) = 1\;\Rightarrow\;
    P(D\mid I) = \sum_i P(D\mid M_i, I)\,P(M_i\mid I).
  \end{align}
\item $P(M_i \mid I)$ is how true we thought model $M_i$ is a priori,
  the prior probability for model $M_i$.  For example, if a proposed
  model violates a physical conservation law, we know it is unlikely
  to be true even before we see the data.  In practice, we assign
  equal probability to all models we have not ruled out prior to seeing
  the data.  I.e., we have uninformative priors for the models.
\item $P(D\mid M_i, I)$ is the likelihood of observing the data, given
  that model $M_i$ is true.
\end{itemize}

As usual, we need to specify the likelihood and prior to assess the
posterior probability of any given model.  We already discussed how to
specify the prior.  We usually assume all models are equally likely.
How about the likelihood?  Well, glancing at equation
\eqref{eq:param_est}, we see that the likelihood for the model
selection problem is the evidence for the parameter estimation
problem!  Because the posterior in the parameter estimation problem,
$P(\mathbf{a}_i\mid D, M_i, I)$, must be normalized, the evidence in
the parameter estimation problem, and therefore the likelihood in the
model selection problem, is given by
\begin{align}
P(D\mid M_i, I) = \int \mathrm{d}\mathbf{a}_i\,P(D\mid \mathbf{a}_i, M_i, I)\,
P(\mathbf{a}_i\mid M_i, I).
\label{eq:bayes_factor_integral}
\end{align}
So, if we can compute the likelihood and priors from the parameter
estimation problem and can integrate their product, we have the
likelihood for the model selection problem.


% %%%%%%%%%%%%%
\subsection{Bayes factors and odds ratios}
Computing the absolution probability of a model is difficult, since it
would require considering all possible models, as is required to
compute the normalization constant, $P(D\mid I)$.  We typically
therefore make pairwise comparisons between models.  This comparison
is called an \textbf{odds ratio}.  It is the ratio of the
probabilities of two models being true.
\begin{align}
O_{ij} = \frac{P(M_i\mid I)}{P(M_j\mid I)} 
\left[\frac{P(D\mid M_i, I)}{P(D\mid M_j, I)}\right].
\end{align}
The first factor in the product is the ratio of our prior knowledge of
the truth of the models.  If they are equally likely, this ratio is
unity.  The bracketed ratio is called the \textbf{Bayes factor}, which
is what the data told us about the truth of the respective models.

Note that if we compute all of the odds ratios comparing a given model
$k$ to all other (and somehow did manage to consider all models that
have nonzero probability), we can compute the posterior probability of
model $M_i$ as
\begin{align}
P(M_i\mid D, I) = \frac{O_{ik}}{\sum_j O_{jk}}.
\end{align}


% %%%%%%%%%%%%%
\subsection{Approximate computation of the Bayes factor}
Evaluating the integral in equation \eqref{eq:bayes_factor_integral} to
compute the Bayes factor is in general difficult.  If the posterior is
sharply peaked, we may compute this integral using the
\textbf{Laplace approximation} in which we approximated the integral
by the height of the peak times its width.  In one dimension, this is
\begin{align}
\int \mathrm{d}a_i\,P(D\mid a_i, M_i, I)\,P(a_i\mid M_i, I)
\approx P(D\mid a_i^*, M_i, I)\,P(a_i^*\mid M_i, I)\,\sqrt{2\pi\sigma_i^2},
\end{align}
where $a^*$ is the MAP estimate, $\sigma_i^2$ is the variance of the
Gaussian approximation of the posterior.  In $n$-dimensions, this is
\begin{align}
\int \mathrm{d}a_i\,P(D\mid \mathbf{a}_i, M_i, I)\,P(\mathbf{a}_i\mid M_i, I)
\approx P(D\mid \mathbf{a}_i^*, M_i, I)\,P(\mathbf{a}_i^*\mid M_i, I)\,\left(2\pi\right)^{|\mathbf{a}_i|/2}\sqrt{\det\boldsymbol{\sigma}_i^2},
\end{align}
where $\boldsymbol{\sigma}^2$ is now the covariance matrix of the
Gaussian approximation of the posterior.  Note that we have already
computed all of factors in the product in the parameter estimation
problem.  Therefore, we already have what we need to compute the odd
ratio.


% %%%%%%%%%%%%%
\subsection{The factors in the odds ratio}
We can now write the approximate odds ratio as the product of three
factors.
\begin{align}
O_{ij} \approx \left(\frac{P(M_i\mid I)}{P(M_j\mid I)}\right)
\left(\frac{P(D\mid \mathbf{a}_j^*, M_j, I)}{P(D\mid \mathbf{a}_j^*, M_j, I)}\right)
\left(\frac{P(\mathbf{a}_i^*\mid M_j, I)\,\left(2\pi\right)^{|\mathbf{a}_i|/2}\sqrt{\det\boldsymbol{\sigma}_j^2}}
{P(\mathbf{a}_j^*\mid M_j, I)\,\left(2\pi\right)^{|\mathbf{a}_j|/2}\sqrt{\det\boldsymbol{\sigma}_j^2}}\right).
\end{align}
\begin{itemize}
\item The first term is our prior probability of the models.  This is
  how plausible we thought the models were before the experiment.
\item The second term is a measure of the goodness of fit.  In other
  words, it comments on how probably the data are given model and the
  MAP estimate.
\item The third term is a ratio of \textbf{Occam factors}.  An Occam
  factor is the ratio of the volume of parameter space accessible to
  the posterior to that of the prior.  This is best seen by example.
  Consider a single parameter model where the parameter has a uniform
  prior.  Then,
  \begin{align}
    \text{Occam factor} \propto P(a\mid M_i, I) \sigma_i = \frac{\sigma_i}{a_\mathrm{max} - a_\mathrm{min}}.
  \end{align}
  Now, compare a model with one parameter ($a$) with uniform prior to
  one with two ($a$ and $b$).  In this case, we have
  \begin{align}
    P(a^*\mid M_1,I) &= \frac{1}{a_\mathrm{max} - a_\mathrm{min}}, \\
    P(a^*, b^*\mid M_2,I) &= \frac{1}{a_\mathrm{max} - a_\mathrm{min}}\,
                            \frac{1}{b_\mathrm{max} - b_\mathrm{min}}.
  \end{align}
  So, the volume of the parameter space model $M_2$ is larger than
  $M_1$, so the this part of the odds ratio is greater than one,
  favoring the model with fewer parameters.  The ratio of Occam
  factors is then
  \begin{align}
    \frac{\sigma_i}{\sqrt{2\pi \,\det \boldsymbol{\sigma}_j^2}}\, (b_\mathrm{max} - b_\mathrm{min}).
  \end{align}
  It is also often the case that complicated models with lots of
  parameters also have smaller determinant of the covariance because
  the multitude of parameters are ``locked in'' around the MAP
  estimate.  Thus, we see where the Occam factor gets its name, since
  it it penalizes more complicated models.\footnote{Remember that
    Occam's razor states that among competing hypotheses, the one with
    fewest assumptions is preferred.}
\end{itemize}

This approximate calculation shows us everything that goes into the
odds ratio.  Any one factor can overwhelm the others: What we knew
before, how well the model fits the data, and how simple the model is.


% %%%%%%%%%%%%%
\subsection{Example: Are two data sets from the same distribution?}
We will now look at an example.  Say I do two sets of measurements of
property $x$, a control and an experiment.  We make $n_c$ control
measurements and $n_e$ experiment measurements.  We consider two
models.  Model $M_1$ says that both the control and the experiment are
chosen from the same underlying Gaussian distribution with mean $\mu$
and variance $\sigma$.  Model $M_2$ says that control and experiment
come from different Gaussian distributions with means $\mu_c$ and
$\mu_e$.  We wish to compare models $M_1$ and $M_2$.  The odds ratio is
\begin{align}
O_{12} = \frac{P(M_1\mid I)}{P(M_2\mid I)} 
\frac{P(D_c,D_e\mid M_1, I)}{P(D_c,D_e\mid M_2, I)},
\end{align}
where $D_c$ denotes the data from the control experiment and $D_e$
denotes the data from the experiment.

We will assume a prior that $P(M_i\mid I) = P(M_j \mid I)$.  Then, we
are left to compute $P(D_c,D_e\mid M_1, I)$ and
$P(D_c,D_e\mid M_2, I)$.  We can do this by approximate macho
integration (see section 4.3.1 of Sivia).  Note that we assume a
uniform prior on $\sigma$, with $0 < \sigma < \sigma_\mathrm{max}$.
We could also try the problem with a Jeffreys prior on $\sigma$, but I
do not feel like doing the macho integration.  The result for the odds
ratio is
\begin{align}
O_{12} \approx \frac{\sigma_\mathrm{max}\left(\mu_\mathrm{max} - \mu_\mathrm{min}\right)}{\pi \sqrt{2}}\;\frac{n_1\,n_2\, s^{2-n1-n2}}{(n_1+n_2)\,s_1^{2-n_1}\,s_2^{2-n_2}},
\end{align}
where
\begin{align}
s^2 &= \frac{1}{n_1+n_2}\sum_{i\in D_1 \cup D_2} (x_i - \bar{x})^2, \\
s_1^2 &= \frac{1}{n_1}\sum_{i\in D_1} (x_i - \bar{x}_1)^2, \\
s_2^2 &= \frac{1}{n_2}\sum_{i\in D_2} (x_i - \bar{x}_2)^2,
\end{align}
with
\begin{align}
  \bar{x} &= \frac{1}{n_1+n_2}\sum_{i\in D_1 \cup D_2} x_i,\\
  \bar{x}_1 &= \frac{1}{n_1}\sum_{i\in D_1} x_i,\\
  \bar{x}_2 &= \frac{1}{n_2}\sum_{i\in D_2} x_i.
\end{align}

It seems that this question is often asked: does the experiment come
from a different process than the control?  My opinion is that in most
situations, the answer is an obvious yes, and the more pertinent
question is by how much they differ.  Nonetheless, if we are asking
the ``if they are different'' question, we can plug our data in and
easily compute it.
