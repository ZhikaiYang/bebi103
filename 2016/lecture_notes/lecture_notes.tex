\documentclass [12pt,letter]{article}
\usepackage{setspace}
\usepackage{amssymb}
\usepackage{amsmath}
\usepackage{amsfonts}
\usepackage{amssymb}
\usepackage{setspace}
% \usepackage{citesort}
\usepackage{amsthm}
\usepackage{textcomp}
\usepackage{graphicx}
\usepackage{url}
\usepackage{color}
\usepackage{comment}
\usepackage[colorlinks=true, urlcolor=blue, hyperfootnotes=false, hyperfigures=false, hyperindex=false, linkcolor=black]{hyperref}

\setlength{\evensidemargin}{0.0cm}
\setlength{\oddsidemargin}{0.0cm}
\setlength{\topmargin}{0.0cm}
%\setlength{\baselineskip}{20pt}
\setlength{\textwidth}{17cm}
\setlength{\textheight}{20cm}
\setlength{\parskip}{2.5mm}

% Give more spacing in equation arrays
\setlength{\jot}{10pt}

% Allow page breaks in multiline equations
\allowdisplaybreaks

% Fancy lines to separate text
\newcommand{\nicehrule}{\begin{center}
    \rule[2pt]{0.3\linewidth}{1pt}$\,\star\star\star\,$\rule[2pt]{0.3\linewidth}{1pt}
    \end{center}}


% Theorems, lemmas, definitions, and problems
\theoremstyle{plain}
\newtheorem{theorem}{Theorem}[section]

\theoremstyle{plain}
\newtheorem{lemma}{Lemma}[section]

\theoremstyle{definition}
\newtheorem{definition}{Definition}[section]

\theoremstyle{definition}
\newtheorem{problem}{Problem}[section]

% Number equations within sections
\numberwithin{equation}{section}

% %%%%%%%%%%%%%%%%%%%%%%%%%%%%%%%%%%%%%%%%%%%%%%%%%%%%%%%%%%%%%%%%%%%%%%%
% %%%%%%%%%%%%%%%%%%%%%%%%%%%%%%%%%%%%%%%%%%%%%%%%%%%%%%%%%%%%%%%%%%%%%%%
% This is where we define whether or not we will show the solutions
% when we compile.  To show the solutions, use
% \includecomment{solution}.  To hide the solutions, use
% \excludecomment{solution}

\includecomment{solution}
%\excludecomment{solution}
% %%%%%%%%%%%%%%%%%%%%%%%%%%%%%%%%%%%%%%%%%%%%%%%%%%%%%%%%%%%%%%%%%%%%%%%
% %%%%%%%%%%%%%%%%%%%%%%%%%%%%%%%%%%%%%%%%%%%%%%%%%%%%%%%%%%%%%%%%%%%%%%%


\begin{document}

% %%%%%%%%%%%%%%%%%%%%%%%%%%%%%%%%%%%%%%%%%%%%%%%%%%%%%%%%%%%%%%%%%%%%%%%%
% \section{Template}
% \input{template}
% %%%%%%%%%%%%%%%%%%%%%%%%%%%%%%%%%%%%%%%%%%%%%%%%%%%%%%%%%%%%%%%%%%%%%%%%

\vspace{10cm}
\begin{center}
\Huge{\textbf{BE/Bi 103: Data Analysis in the Biological Sciences}} \\ \vspace{2mm}
\Large{Justin Bois} \\ \vspace{2mm}
\Large{Caltech} \\ \vspace{2mm}
\Large{Fall, 2016} \\

\vspace{15em}

\footnotesize{\noindent{\textcopyright~2016 Justin Bois. This work is licensed under a \href{https://creativecommons.org/licenses/by/4.0/}{Creative Commons Attribution License CC-BY 4.0}.}}
\end{center}

\thispagestyle{empty}

\pagebreak
\setcounter{page}{1}

% %%%%%%%%%%%%%%%%%%%%%%%%%%%%%%%%%%%%%%%%%%%%%%%%%%%%%%%%%%%%%%%%%%%%%%%
\section{Bayes's theorem and the logic of science}
\label{sec:bayes_logic}
We start with a question.  \textbf{What is the goal of doing
  (biological) experiments?}  There are many answers you may have for
this.  Some examples:
\begin{itemize}
\item To further knowledge.
\item To test a hypothesis.
\item To explore and observe.
\item To demonstrate, e.g., demonstrate feasibility.
\end{itemize}

More obnoxious answers are
\begin{itemize}
\item To graduate.
\item Because your PI said so.
\item To get data.
\end{itemize}

This question might be better addressed if we zoom out a bit and think
about the scientific process as a whole.  In
Fig.~\ref{fig:scientific_process}, we have a sketch of the scientific
processes.  This cycle repeats itself as we explore nature and learn
more.  In the boxes are milestones, and along the arrows in red text
are the tasks that get us to these milestones.

\begin{figure}[h]
\centerline{
        \includegraphics[width=0.9\linewidth]{figs/scientific_process.pdf}}
      \caption{A sketch of the scientific process.  Adapted from
        Fig. 1.1 of P. Gregory, \textit{Bayesian Logical Data Analysis
          for the Physical Sciences}, Cambridge, 2005.}
\label{fig:scientific_process}
\end{figure}

Let's consider the tasks and milestones.  We start in the lower left.

\begin{itemize}
\item \textit{Hypothesis invention/refinement}.  In this stage of the
  scientific process, the researcher(s) think about nature, all that
  they have learned, including from their own previous experiments,
  and formulate hypotheses or theories they can pursue with
  experiments.  This step requires innovation, and sometimes genius
  (e.g., general relativity).
\item \textit{Deductive inference}. Given the hypothesis, the
  researchers deduce what must be true if the hypothesis is true.  You
  have done a lot of this in your study to this point, e.g.,
  \textit{given $X$ and $Y$, derive $Z$}.  Logically, this requires a
  series of \textbf{strong syllogisms}:\\
  \phantom{blahblah}If $A$ is true, then $B$ is true.\\
  \phantom{blahblah}A is true.\\
  \phantom{blahblah}Therefore B is true.\\
  The result of deductive inference is a set of (preferably
  quantitative) predictions that can be tested experimentally.
\item \textit{Do experiment}. This requires \textit{work}, and also
  its own kind of innovation.  Specifically, you need to think
  carefully about how to construct your experiment to test the
  hypothesis. It also usually requires money.  The result of doing
  experiments is data.  
\item \textit{Statistical (plausible) inference}. This step is perhaps
  the least familiar to you, but \textit{this is the step that this
    course is all about}.  I will talk about what statistical
  inference is next; it's too involved for this bullet point.  But the
  result of statistical inference is knowledge about how
  \textit{plausible} a hypothesis and estimates of parameters under
  that hypothesis.
\end{itemize}


% %%%%%%%%%%%%%%%%
\subsection{What is statistical inference?}
As we designed our experiment under our hypothesis, we used deductive
logic to say, ``If $A$ is true, then $B$ is true,'' where $A$ is our
hypothesis and $B$ is an experimental observation.  This was
\textit{deductive} inference.

Now, let's say we observe $B$.  Does this make $A$ true?  Not
necessarily.  But it does make $A$ more \textit{plausible}.  This is
called a \textit{weak syllogism}.  As an example, consider the
following hypothesis/observation pair.
\begin{align*}
&A = \text{California is at the beginning of a megadrought.}\\
&B = \text{2000-2014 were the driest years since California became a state.}\\
&\text{Because } B \text{ was observed, } A \text{ is more plausible.}
\end{align*}

This is all fine and good, but we need a way to decide \textit{how
  much more plausible} $A$ is.  In other words, we need a way to
quantify plausibility.

So, \textbf{statistical inference requires a probability theory.}
Thus, probability theory is a generalization of logic.  Due to this
logical connection and its crucial role in science, E. T. Jaynes says
that probability is the ``logic of science.''


% %%%%%%%%%%%%%%%
\subsection{The problem of probability}
We know what we need, a theory called probability to quantify
plausibility.  We will not formally define probability in class, but
use our common sense reasoning of it.  Nonetheless, it is important to
understand that there are two dominant interpretations of probability.

\paragraph{Frequentist probability.}  In the \textit{frequentist}
definition of probability, the probability $P(A)$ represents a
long-run frequency over a large number of identical repetitions of an
experiment.  These repetitions can be, and often are, hypothetical.
The event $A$ is restricted to propositions about \textit{random
  variables}, a quantity that can very meaningfully from experiment to
experiment.\footnote{More formally, a random variable transforms the
  possible outcomes of an experiment to real numbers.}

\paragraph{Bayesian probability.} Here, $P(A)$ directly represents the
degree of belief, or plausibility, about $A$.  So, $A$ can be any
logical proposition.

You may have heard about a split, or even a fight, between people who
use Bayesian statistics and frequentist statistics.  There is no need
for a fight.  The two ways of approaching statistical inference differ
in their definition of probability, the tool we use to quantify
plausibility.  Either is valid.

In my opinion, the Bayesian definition of probability is more
intuitive to apply to scientific inference.  It always starts with a
simple probabilistic expression and proceeds to quantify plausibility.
It is conceptually cleaner to me, since we can talk about plausibility
of anything, including parameter values.  In other words, Bayesian
probability serves to quantify our own knowledge, or degree of
certainty, about a hypothesis or parameter value.  Conversely, in
frequentist statistical inference, the parameter values are fixed, and
we can only study how repeated experiments will convert the real value
to a real number.

We will learn about some frequentist approaches in class, but we will
generally focus on Bayesian analysis.

% %%%%%%%%%%%%%%%
\subsection{Desiderata for Bayesian probability}
In 1946, R. Cox laid out a pair of rules based on some desired
properties of probability as a quantifier of plausibility.  These ideas
were expanded on by E. T. Jaynes in the 1970s.  The
\textit{desiderata} are
\begin{itemize}
\item[I.] Probability is represented by real numbers.
\item[II.] Probability must agree with rationality.  As more
  information is supplied, probability must rise in a continuous,
  monotonic manner.  The deductive limit must be obtained where
  appropriate.
\item[III.] Probability must be consistent.
  \begin{enumerate}
  \item[a)] Structure consistency: If a result is reasoned in more
    than one way, we should get the same result.
  \item[b)] Propriety: All relevant information must be considered.
  \item[c)] Jaynes consistency: Equivalent states of knowledge must be
    represented by equivalent probability.
  \end{enumerate}
\end{itemize}

Two results of these desiderata (worked out in chapter 2 of Gregory's
book) are the \textit{sum rule} and the \textit{product rule}.

% %%%%%%%%%%%%%%%
\subsection{The sum rule, the product rule, and conditional probability}
The \textit{sum rule} says that the probability of all events must add
to unity.  Let $\bar{A}$ be all events \textit{except} $A$.  Then, the
sum rule states that
\begin{align}
  P(A) + P(\bar{A}) = 1.
\end{align}

Now, let's say that we are interested in event $A$ happening
\textit{given} that even $B$ happened.  So, $A$ is
\textit{conditional} on $B$.  We denote this conditional probability
as $P(A\mid B)$.  Given this notion of conditional probability, we can
write the sum rule as
\begin{align}
  \text{\textbf{(sum rule)}} \qquad P(A\mid B) + P(\bar{A} \mid B) = 1,
\end{align}
for any $B$.

The \textit{product rule} states that
\begin{align}
  P(A, B) = P(A\mid B)\, P(B),
\end{align}
where $P(A,B)$ is the probability of both $A$ \textit{and} $B$
happening.  The product rule is also referred to as the definition of
conditional probability.  It can similarly be expanded as we did the
the product rule.
\begin{align}
   \text{\textbf{(product rule)}} \qquad P(A, B\mid C) = P(A\mid B, C)\, P(B \mid C),
\end{align}
for any $C$.

% %%%%%%%%%%%%%%%
\subsection{Application to scientific measurement}
This is all a bit abstract.  Let's bring it into the realm of
scientific experiment.  We'll assign meanings to these things we have
been calling $A$, $B$, and $C$.
\begin{align}
A &= \text{hypothesis (or parameter value), } H_i, \\
B &= \text{Measured data set, } D,\\
C &= \text{All other information we know, } I.
\end{align}
Now, let's rewrite the product rule.
\begin{align}
P(H_i, D\mid I) = P(H_i \mid D, I)\, P(D \mid I).
\end{align}
Ahoy!  The quantity $P(H_i \mid D , I)$ is exactly what we want from
our statistical inference.  This is the probability that a hypothesis
is true, or a probability distribution for the values of a parameter,
given measured data and everything we've learned.  Now, how do we
compute it?


% %%%%%%%%%%%%%%%
\subsection{Bayes's Theorem}
Note that because ``and'' is commutative,
$P(H_i, D \mid I) = P(D, H_i \mid I)$.  So, we just apply the product
rule to both sides of the seemingly trivial equality.
\begin{align}
  P(H_i \mid D, I)\, P(D \mid I) =  P(H_i, D \mid I) 
  = P(D, H_i \mid I) = P(D \mid H_i, I)\, P(H_i \mid I).
\end{align}
If we take the terms at the beginning and end of this equality and
rearrange, we get
\begin{align}
\text{\textbf{(Bayes's theorem)}} \qquad  P(H_i \mid D, I) = \frac{P(D \mid H_i, I)\, P(H_i \mid I)}{P(D \mid I)}.
\end{align}
This result is called \textbf{Bayes's theorem}.  This is far more
instructive in terms of how to compute our goal, which is the left
hand side.  The quantities on the right hand side all have meaning.
We will talk about the meaning of each term in turn, and this is
easier to do using their names.  Each item in Bayes's theorem has a
name.

\begin{align}
\text{posterior} = \frac{\text{likelihood} \times \text{prior}}{\text{evidence}}.
\end{align}

\paragraph{The prior probability.}  First, consider the prior,
$P(H_i \mid I)$.  As probability is a measure of plausibility, or how
believable a hypothesis is, we should be able to write this down based
on $I$.\footnote{I say this flippantly.  In fact, specifying prior
  probabilities is one of the most studied and most controversial
  aspects of Bayesian statistics.}  This represents the
plausibility about hypothesis $H_i$ given everything we know
\textit{before} we did the experiment to get the data.

\paragraph{The likelihood.}
The likelihood, $P(D\mid H_i,I)$, describes how likely it is to
acquire the observed data, \textit{given that the hypothesis} $H_i$
\textit{is true}.  It also contains information about what we expect
from the data, given our measurement method.  Is there noise in the
instruments we are using?  How do we model that noise?  These are
contained in the likelihood.

\paragraph{The evidence.}  I will not talk much about this here,
except to say that it can be computed from the likelihood and prior,
and is also called the \textit{marginal likelihood}, a name whose
meaning will become clear in the next section.

\paragraph{The posterior probability.} This is what we are after.  How
plausible is the hypothesis, given that we have measured some new
data?  It is calculated directly from the likelihood and prior (since
the evidence is also computed from them).  Computing the posterior
distribution constitutes the bulk of our tasks in this course.


% %%%%%%%%%%%%%%%
\subsection{Marginalization}
A moment ago, I mentioned that the evidence can be computed from the
likelihood and the prior.  To see this, we apply the sum rule to the
posterior probability.
\begin{align}
1 &= P(H_j\mid D,I) + P(\bar{H}_j | D,I) \\
&= P(H_j\mid D,I) + \sum_{i\ne j}P(H_i\mid D,I) \\
&= \sum_iP(H_i\mid D,I),
\end{align}
for some hypothesis $H_j$.  Now, Bayes's theorem gives us an
expression for $P(H_i\mid D, I)$, so we can compute the sum.
\begin{align}
\sum_iP(H_i\mid D,I) = \sum_i\frac{P(D \mid H_i, I)\, P(H_i \mid I)}{P(D \mid I)}
= \frac{1}{P(D\mid I)}\sum_i P(D \mid H_i, I)\, P(H_i \mid I) = 1.
\end{align}
Therefore, we can compute the evidence by summing over the priors and
likelihoods of all possible hypotheses.
\begin{align}
P(D\mid I) = \sum_i P(D \mid H_i, I)\, P(H_i \mid I).
\end{align}
This process of eliminating a variable (in this case the hypotheses)
from a probability by summing is called \textit{marginalization}.

Note that if the space of hypotheses is continuous, e.g., if the
``hypothesis'' is a parameter value which we'll call $\theta$, we can
replace the summation with an integral.\footnote{There are some
  mathematical subtleties.  These are discussed at length in Jaynes's
  book, \textit{Probability Theory: the logic of science}.}
\begin{align}
P(D\mid I) = \int \mathrm{d}\theta\,P(D\mid \theta, I)\, P(\theta \mid I).
\end{align}

% %%%%%%%%%%%%%%%
\subsection{A note on the word ``model''}
You may have noticed the terms ``model 1,'' ``model 2,'' and ``model
3'' in Fig.~\ref{fig:scientific_process}.  Being biologists who are
doing data analysis, the word ``model'' is used to mean three
different things in our work.  So, for the purposes of this course, we
need to clearly define what we are talking about when we use the work
``model.''

\paragraph{Model 1.}  These models are the typical cartoons we see in
text books or in discussion sections of biological papers.  They are a
sketch of what we think might be happening in a system of interest,
but they do not provide quantifiable predictions.

\paragraph{Model 2.} These models give quantifiable predictions that
must be true if the hypothesis (which is sketched as a ``model 1'') is
true. In many cases, getting to predictions from a hypothesis is easy.
For example, if I hypothesis that protein A binds protein B, a
quantifiable prediction would be that they are colocalized when I
image them.  However, sometimes harder work and deeper thought is
needed to generate quantitative predictions.  This often requires
``mathematizing'' the cartoon.  This is how a model 2 is derived from
a model 1.  Oftentimes when biological physicists refer to a
``model,'' they are talking about model 2.

\paragraph{Model 3.} Essentially, model 3 specifies the likelihood.
Statisticians often use the word ``model'' to describe model 3.  As a
simple example, consider the measurement of the length of a
\textit{C. elegans} eggs.  A plausible model 3 would be that the egg
lengths are Gaussian distributed (and therefore are described by a
mean and a standard deviation).  The model 3 can include any
mathematization of cartoons we did to generate model 2, and can also
contain any information about any possible effects we might see in a
measurement.


% %%%%%%%%%%%%%%%
\subsection{Bayes's theorem as a model for learning}
We will close today's lecture with a discussion of Bayes's theorem as
as model for learning.  Let's say we did an experiment and got data
set $D_1$ as a test of hypothesis $H$.  Then, our posterior
distribution is
\begin{align}
P(H\mid D_1, I) = \frac{P(D_1 \mid H, I)\, P(H \mid I)}{P(D_1 \mid I)}.
\label{eq:l01_bayes}
\end{align}
Now, let's say we did another experiment and got data $D_2$.  We
already know $D_1$ ahead of this experiment, do our prior is
$P(H\mid D_1, I)$, which is the posterior from the first experiment.
So, we have
\begin{align}
  P(H\mid D_1, D_2, I) = \frac{P(D_2 \mid D_1, H, I)\, P(H \mid D_1, I)}{P(D_2 \mid D_1, I)}.
\end{align}
Now, we plug in Bayes's theorem applied to our first data set,
equation \eqref{eq:l01_bayes}, giving
\begin{align}
P(H\mid D_1, D_2, I) = \frac{P(D_2 \mid D_1, H, I)\,P(D_1 \mid H, I)\, P(H \mid I)}{P(D_2 \mid D_1, I)\, P(D_1 \mid I)}.
\label{eq:l01_combined}
\end{align}
By the product rule, the denominator is $P(D_1, D_2 \mid I)$.  Also by
the product rule,
\begin{align}
P(D_2 \mid D_1, H, I)\,P(D_1 \mid H, I) = P(D_1, D_2 \mid H, I).
\end{align}
Inserting these expressions into equation \eqref{eq:l01_combined}
yields
\begin{align}
  P(H\mid D_1, D_2, I) = \frac{P(D_1, D_2 \mid H, I)\,P(H\mid I)}{P(D_1, D_2 \mid I)}.
\end{align}
So, acquiring more data gave us more information about our hypothesis,
as if we just combined $D_1$ and $D_2$ into a single data set.

% %%%%%%%%%%%%%%%%%%%%%%%%%%%%%%%%%%%%%%%%%%%%%%%%%%%%%%%%%%%%%%%%%%%%%%%%

\pagebreak

% %%%%%%%%%%%%%%%%%%%%%%%%%%%%%%%%%%%%%%%%%%%%%%%%%%%%%%%%%%%%%%%%%%%%%%%
\section{Parameter estimation from repeated measurements}
\label{sec:parameter_estimation}
In the last lecture, we learned about Bayes's theorem as a way to
update a hypothesis in light of new data.  We use the word
``hypothesis'' very loosely here.  Remember, in the Bayesian view,
probability can describe the plausibility of any proposition.  The
value of a parameter is such a proposition.  In this lecture, we will
learn about how to do a Bayesian estimate of a parameter.


% %%%%%%%%%%%%%%%%%%%
\subsection{Bayes's theorem as applied to simple parameter estimation}
We will consider one of the simplest examples of parameter estimation.
Let's say we measure a parameter $\mu$ in multiple independent
experiments.  This could be beak depths of finches, fluorescence
intensity on a cell, dissociation constant for two bound proteins,
etc.  The possibilities abound.  You can have whatever your favorite
measurement is in mind during this analysis.

Our measurements of this parameter are $D = \{x_1, x_2, \ldots x_n\}$.
Our ``hypothesis'' in this case, is the value of the parameter $\mu$.
So, we wish to calculate $P(\mu \mid D, I)$, the posterior probability
distribution for the parameter $\mu$, given the data.  Values of $\mu$
for which the posterior probability is high are more probable (that
is, more plausible) than those for which is it low.

To compute the posterior probability, we use Bayes's theorem.
\begin{align}
P(\mu\mid D, I) = \frac{P(D\mid \mu, I)\,P(\mu \mid I)}{P(D\mid I)}.
\end{align}
Since the evidence, $P(D\mid I)$ does not depend on the parameter of
interest, $\mu$, it is really just a normalization constant, so we do
not need to consider it explicitly.  We now have to specify the
likelihood $P(D\mid \mu, I)$ and the prior $P(\mu \mid I)$.


% %%%%%%%%%%%%%%%%%%%
\subsection{The likelihood}
To specify the likelihood, we have to ask what we expect from the
data, given a value of $\mu$.  If there are no errors or confounding
factors at all in our measurements, we expect $x_i = \mu$ for all $i$.
In this case
\begin{align}
P(D\mid \mu, I) = \prod_{i\in D}\delta(x_i - \mu),
\end{align}
the product of Dirac delta functions.  Of course, this is really never
the case.  There will be some errors in measurement and/or the system
has variables that confound the measurement.  What, then should we
choose for our likelihood?

This question is made sharper if we think about the likelihood in
terms of the \textit{statistical model} we defined in the last lecture.  It is the
probability distribution that describes how the data relate to the
parameter we are trying to measure. Indeed, specifying the likelihood is part of the modeling process. In \href{http://bebi103.caltech.edu/2016/tutorials/t3a_probability_distributions.html}{Tutorial 3a}, we will
learn more about probability distributions, but for now, we will
introduce one useful distribution to use in our analyses.

% %%%%%%%%%%%%%%%%%%%
\subsection{The Gaussian distribution}
A Gaussian, or Normal, probability distribution has a probability
density function (PDF) of
\begin{align}
  P(x \mid \mu, \sigma) = \frac{1}{\sqrt{2\pi\sigma^2}}\,
\exp\left\{-\frac{(x - \mu)^2}{2\sigma^2}\right\}.
\end{align}
The parameter $\mu$ is called the mean of the distribution and
$\sigma^2$ is the variance, with $\sigma$ being called the standard
deviation. Importantly, the mean and standard deviation in this context are \textit{names of parameters} of the distribution; they are not what you compute directly from data.

The \textbf{central limit theorem} says that any quantity that emerges
from a large number of subprocesses tends to be Gaussian distributed,
provided none of the subprocesses is very broadly distributed.  We
will not prove this important theorem, but we will make use of it when
choosing likelihood distributions.

Indeed, in the simple case of estimating a single parameter where many
processes may contribute to noise in the measurement, the Gaussian
distribution is a good choice for a likelihood.\footnote{It is also the
  \textbf{maximal entropy distribution} for a system with well-defined
  first and second moments, but we will not talk about entropy in this
  class.}

More generally, the multi-dimensional Gaussian distribution for
$\mathbf{x} = (x_1, x_2, \cdots, x_n)$ is
\begin{align}
  P(\mathbf{x} \mid \mu, \boldsymbol{\sigma}) = (2\pi)^{-\frac{n}{2}} \left(\det \boldsymbol{\sigma}^2\right)^{-\frac{1}{2}}\,
  \exp\left\{-\frac{1}{2}(\mathbf{x} - \mu)^T\cdot (\boldsymbol{\sigma}^2)^{-1}\cdot(\mathbf{x} - \mu)\right\},
\end{align}
where $\boldsymbol{\sigma}^2$ is a symmetric positive definite matrix
called the \textbf{covariance matrix}.  If off-diagonal entry
$(\boldsymbol{\sigma}^2)_{ij}$ is nonzero, then $x_i$ and $x_j$ are
correlated.  In the case where all $x_i$ are independent, all
off-diagonal terms in the covariance matrix are zero, and the
multidimensional Gaussian distribution reduces to
\begin{align}
  P(\mathbf{x} \mid \mu, \boldsymbol{\sigma}) = \prod_{i=1}^n \frac{1}{\sqrt{2\pi \sigma_i^2}}\,
  \exp\left\{-\frac{(x_i - \mu)^2}{2\sigma_i^2}\right\},
\end{align}
where $\sigma^2_i$ is the $i$th entry along the diagonal of the
covariance matrix.  This is the variance associated with measurement
$i$.  So, if all independent measurements have the same variance, the
multi-dimensional Gaussian reduces to
\begin{align}
P(\mathbf{x} \mid \mu, \sigma) = \left(\frac{1}{2\pi \sigma^2} \right)^{-\frac{n}{2}}\,
  \exp\left\{-\frac{1}{2\sigma^2}\,\sum_{i=1}^n (x_i - \mu)^2\right\}.
\end{align}



% %%%%%%%%%%%%%%%%%%%
\subsection{The likelihood revisited: and another parameter}
The Gaussian distribution is a good choice for our likelihood for
repeated measurements, and we will use it for the likelihood for our
present problem of estimating a parameter form repeated measurements.
We have to decide how the measurements are related to specify how many
entries in the covariance matrix we need to specify as parameters.  It
is often the case that the measurements are independent and identically
distributed (i.i.d.), so that only a single variance, $\sigma^2$, is
specified.  So, we choose our likelihood to be
\begin{align}
  P(D\mid \mu, \sigma, I) = \left(\frac{1}{2\pi \sigma^2} \right)^{\frac{n}{2}}\,
  \exp\left\{-\frac{1}{2\sigma^2}\,\sum_{i\in D} (x_i - \mu)^2\right\},
\end{align}
with $n = |D|$.  By choosing this as our likelihood, we are saying
that we expect our measurements to have a well-defined mean $\mu$ with
a spread described by the variance, $\sigma^2$.

But wait a minute; we now have another parameter, $\sigma$, beyond the
one we're trying to measure.  So, our model (model 3) has \textit{two}
parameters, and Bayes's theorem now reads
\begin{align}
P(\mu, \sigma \mid D, I) = \frac{P(D\mid \mu, \sigma, I)\,P(\mu, \sigma \mid I)}
{P(D\mid I)}.
\end{align}
After we compute the posterior, we can still find the probability
distribution we are after by marginalizing.
\begin{align}
P(\mu\mid D, I) = \int_0^\infty \mathrm{d}\sigma\,P(\mu, \sigma \mid D, I).
\end{align}


% %%%%%%%%%%%%%%%%%%%
\subsection{Choice of prior}
Because the evidence $P(D\mid I)$ is entirely determined by the
likelihood, prior, and normalization condition of the posterior, we
need only to specify the likelihood and prior to get the posterior.
We have chosen a Gaussian distribution for our likelihood, so now we
need to specify $P(\mu, \sigma \mid I)$.  The prior encodes what we
know about the parameters \textit{before} the experiments.  The prior
may be informed by previous experiments, as we discussed in section
\ref{sec:l01_learning}.  But what if we know nothing?

One assumption we could make is that $\mu$ and $\sigma$ are
independent.  In this case,
\begin{align}
P(\mu, \sigma \mid I) = P(\mu \mid I)\,P(\sigma\mid I).
\end{align}
Now, we have to decide on prior probabilities for $\mu$ and $\sigma$.
Our goal is to choose \textit{uninformative priors}, i.e., we want to
claim maximal ignorance in our choice of prior probability when we
have to prior information.

For $\mu$, we will choose a uniform prior, meaning that all values are
equally likely.
\begin{align}
P(\mu\mid I) = \left\{\begin{array}{ccl}
\left(\mu_\mathrm{max} - \mu_\mathrm{min}\right)^{-1} & & \mu_\mathrm{min} < \mu < \mu_\mathrm{max}, \\[1em]
0 & & \text{otherwise}
\end{array}\right.
\end{align}
We have put bounds on the allowed values of $\mu$.  You may argue that
this is informative; we have chosen bounds.  We intentionally choose
the bounds to be so far away from the peak of the likelihood that the
posterior is vanishingly small at the bounds.  Thus, the parameters
$\mu_\mathrm{min}$ and $\mu_\mathrm{max}$ are arbitrary and serve only
to ensure the $P(\mu \mid I)$ is a proper probability
distribution.\footnote{There is really no problem with it not being
  proper.  In fact, improper priors are commonly used.}

For the parameter $\sigma$, the choice is not quite so simple.  First,
we know that $\sigma > 0$ by construction.  We also note that we could
have chosen to parametrize the Gaussian distribution with
$\xi \equiv \sigma^{-1}$ instead of $\sigma$.  We want there to be no
bias as a result of this choice.  In other words, we want
\begin{align}
P(\sigma\mid I) = P(\xi \mid I) \, \left|\frac{\mathrm{d}\xi}{\mathrm{d}\sigma}\right|,
\label{eq:l02_jeffreys1}
\end{align}
where we have employed the change of variables
formula.\footnote{Remember your calculus: $f(x) =
  f(y)\left|\mathrm{d}y/\mathrm{d}x\right|$.}  We will choose
\begin{align}
  P(\sigma \mid I) =  \left\{\begin{array}{ccl}
\left(\ln(\sigma_\mathrm{max} / \sigma_\mathrm{min})\,\sigma\right)^{-1} & & \sigma_\mathrm{min} < \sigma < \sigma_\mathrm{max}\\[1em]
0 & & \text{otherwise},
\end{array}\right.
\end{align}
and show that the condition given in equation \eqref{eq:l02_jeffreys1}
hold.  For $\sigma_\mathrm{max}^{-1} < \xi < \sigma_\mathrm{min}^{-1}$, we
have
\begin{align}
P(\xi \mid I) \, \left|\frac{\mathrm{d}\xi}{\mathrm{d}\sigma}\right|
= \ln(\sigma_\mathrm{max} / \sigma_\mathrm{min})\,\xi^{-1} \, \sigma^{-2}
=  \ln(\sigma_\mathrm{max} / \sigma_\mathrm{min})\,\sigma^{-1}
= P(\sigma\mid I),
\end{align}
so the condition holds.  It holds trivially outside of the bounds.
This prior, $P(\sigma\mid I) \propto \sigma^{-1}$, is often called a
\textbf{Jeffreys prior}\footnote{The term \textit{Jeffreys prior} can also be used to describe a class of uninformative priors more generally, but we will not dwell on that here.}, as Harold Jeffreys suggested it in 1939.  It is an uninformative prior to use for
\textbf{scale parameters} like $\sigma$, which can equivalently be
represented by their reciprocals.  Like the prior for $\mu$, the
bounds $\sigma_\mathrm{min}$ and $\sigma_\mathrm{max}$ can be chosen
to be sufficiently large/small such that they are immaterial.

\paragraph{An aside about the Jeffreys prior.} Again using the change
of variables formula, we see that
\begin{align}
P(\sigma \mid I) = P(\ln \sigma \mid I)\left|\frac{\mathrm{d}\ln \sigma}{\mathrm{d}\sigma}\right|
=  \frac{1}{\sigma}\, P(\ln \sigma \mid I).
\end{align}
Thus,
\begin{align}
P(\ln \sigma \mid I) = \left\{\begin{array}{ccl}
\left(\ln \sigma_\mathrm{max} - \ln \sigma_\mathrm{min}\right)^{-1} & & \ln \sigma_\mathrm{min} < \ln \sigma < \ln \sigma_\mathrm{max}, \\[1em]
0 & & \text{otherwise}.
\end{array}\right.
\end{align}
So, if a parameter has a Jeffreys prior, its logarithm has a uniform
prior.  This can be convenient when defining parameters and performing
optimizations.


% %%%%%%%%%%%%%%%%%%%
\subsection{The posterior}
Now that we have specified the likelihood and prior, we have the
posterior.
\begin{align}
P(\mu, \sigma \mid D, I) = \frac{c}{\sigma^{n+1}}\,
\exp\left\{-\frac{1}{2\sigma^2}\,\sum_{i\in D} (x_i - \mu)^2\right\},
\end{align}
where we have absorbed all constants in to the normalization constant
$c$\footnote{We do this here for convenience, but when we do model selection later on, we will have to compute the evidence, so we should be careful about the normalization constants of the priors throughout our calculations.}.  We could do this because we chose the bounds on the priors of
$\mu$ and $\sigma$ to be sufficiently far away from the peak of the
likelihood that they contribute negligibly to the posterior.  This
also allows us to extend our bounds of integration to infinity when
integrating.

So, we are done!  We have now updated our knowledge of $\mu$ and
$\sigma$.  We could just plot the posterior distribution.  We could
show it as a contour plot in the $\mu$-$\sigma$ plane, for instance.

But, it would be nice to get the posterior into a bit of a cleaner
form.  We can show, after some algebraic grunge, that
\begin{align}
\sum_{i\in D}(x_i - \mu)^2 = n(\bar{x} - \mu)^2 + nr^2,
\end{align}
where
\begin{align}
r^2 = \frac{1}{n}\sum_{i\in D}(x_i - \bar{x})^2
\end{align}
is the sample variance and
\begin{align}
\bar{x} = \frac{1}{n}\sum_{i\in D}x_i.
\end{align}
Thus, we have
\begin{align}
P(\mu, \sigma \mid D, I) = \frac{c\,\mathrm{e}^{-nr^2/2\sigma^2}}{\sigma^{n+1}}\,
\exp\left\{-\frac{n(\mu - \bar{x})^2}{2\sigma^2}\right\}.
\end{align}
In this form, we immediately see that, regardless the value of
$\sigma$, the most probable value of $\mu$ is $\bar{x}$.  This is
perhaps not surprising that the most probable value of $\mu$ is the
sample mean, but it is pleasing how nicely it falls out of the
analysis.

Now, it would really like to get a summary of the posterior to be able
to report some nice numbers, like most the probable $\mu = \bar{x}$,
instead of a plot.


% %%%%%%%%%%%%%%%%%%%
\subsubsection{The mean $\mu$}
We wanted to get $P(\mu\mid D, I)$ in the first place.  As we said
before, we can get that by marginalizing over $\sigma$.
\begin{align}
P(\mu\mid D, I) &= \int_0^\infty \mathrm{d}\sigma\,P(\mu, \sigma \mid D, I) \\ \nonumber
&= c\int_0^\infty \frac{\mathrm{d}\sigma}{\sigma^{n+1}}\,
\exp\left\{-\frac{n(\mu - \bar{x})^2 + nr^2}{2\sigma^2}\right\}.
\end{align}
This integral is a little gnarly, but we can evaluate it.  We end up
getting
\begin{align}
P(\mu\mid D, I) \propto \left(1 + \frac{(\mu - \bar{x})^2}{r^2}\right)^{-\frac{n}{2}} \propto \left(\sum_{i\in D} (x_i - \mu)^2\right)^{-\frac{n}{2}}.
\end{align}
I have written the expression in two equivalent forms because it is sometimes more convenient to use one or the other. They are proportional, as you can verify by yourself. For now, we'll use the first expression, since it is convenient for computing the marginalized posteriors.
We can integrate this to get the normalization constant, giving
\begin{align}
P(\mu\mid D, I) = \frac{\Gamma\left(\frac{n}{2}\right)}{\sqrt{\pi}\Gamma\left(\frac{n-1}{2}\right)}\,\frac{1}{r}\,\left(1 + \frac{(\mu - \bar{x})^2}{r^2}\right)^{-\frac{n}{2}}.
\end{align}
The normalization contains gamma functions.  This distribution has a
name.  It is the \textbf{Student-t} distribution.  As we now know, it
describes the estimate mean of a Gaussian distribution with unknown variance from which the data were drawn.  As written, the Student-t
distribution above is said to have $n-1$ degrees of freedom.

As we have already determined, the most probable value of $\mu$ is
$\bar{x}$.  We would like to describe an error bar\footnote{I'm using the term ``error bar'' loosely
  here.  We will sharpen this definition later in the course.} for
this parameter $\mu$.  Since we know its posterior, the error bar is just some summary of the posterior distribution.  We could
report the error bar to contain the set of values of $\mu$,
centered on $\bar{x}$, that contain a given percentage of the
probability.

The common practice for getting the error bar  is to
approximate the posterior distribution as Gaussian and report
intervals based on the standard deviation of the Gaussian
approximation.  To get a Gaussian approximation, we expand the
logarithm of posterior probability distribution function in a Taylor
series about its maximum.
\begin{align}
\ln P(\mu \mid D, I) &= \text{constant} - \frac{n}{2}\,\ln\left(1 + \frac{(\mu - \bar{x})^2}{r^2}\right) \\
&\approx \text{constant} - \frac{n(\mu - \bar{x})^2}{r^2}.
\end{align}
Exponentiating and evaluating the normalization constant yields
\begin{align}
  P(\mu\mid D, I) \approx \frac{1}{\sqrt{2\pi r^2/n}}\,\exp\left\{
-\frac{(\mu - \bar{x})^2}{2r^2/n}
\right\},
\end{align}
a Gaussian distribution with mean $\bar{x}$ and variance $r^2/n$.
Recall that $r^2$ is the sample variance, so the variance of the
Gaussian approximation of the posterior distribution is the sample
variance divided by $n$.  The quantity $r/\sqrt{n}$ is referred to as
the \textbf{standard error of the mean}, which is often how error bars
are reported.  We now know that it describes the width of the
(Gaussian approximation of the) posterior distribution describing the
parameter value we sought to measure.

% %%%%%%%%%%%%%%%%%%%
\subsubsection{The variance $\sigma^2$}
Often overlooked is an estimate for the variance.  Remember, when we
took measurements, we did not assume we knew the variance of the
measurements.  We would also like an estimate of it.

We take a similar approach.  We marginalize the full posterior over
$\mu$.
\begin{align}
P(\sigma\mid D, I) = \int_{-\infty}^\infty \mathrm{d}\mu\,P(\mu,\sigma\mid D, I).
\end{align}
The integral is again doable, but also again a bit gnarly.  The result is
\begin{align}
P(\sigma\mid D, I) = \frac{c}{\sigma^n}\,\exp\left\{-\frac{nr^2}{2\sigma^2}\right\}.
\end{align}
We can compute the normalization constant, but it is a messy
expression including some incomplete gamma functions.  We will not
bother.  Instead, we can find the most probable $\sigma$.  This is
found by finding the value of $\sigma$ for which the derivative of the
posterior is zero.
\begin{align}
  \frac{\mathrm{d}}{\mathrm{d}\sigma}P(\sigma \mid D, I) &=
  c(nr^2\sigma^{-n-3} - n \sigma^{-n-1})\exp\left\{-\frac{nr^2}{2\sigma^2}\right\} \\
  &= c n \sigma^{-n-1}\left(\frac{r^2}{\sigma^2} - 1\right)\exp\left\{-\frac{nr^2}{2\sigma^2}\right\}.
\end{align}
This is zero when $\sigma^2 = r^2$, or
\begin{align}
    \sigma^2 = \frac{1}{n}\sum_{i\in D}(x_i - \bar{x})^2.
\end{align}

We can also compute a confidence interval on the parameter $\sigma$.
Note, though, that its distribution, $P(\sigma \mid D, I)$, is not
symmetric, as seen in Fig.~\ref{fig:l02_sigma_post}.

\begin{figure}[h]
\centerline{
        \includegraphics[width=0.9\linewidth]{figs/sigma_posterior.pdf}}
      \caption{The posterior distribution of $\sigma$ with $r = 1$ for
        various values of $n$.  It becomes more symmetric as $n$ grows.}
\label{fig:l02_sigma_post}
\end{figure}

Given that the distribution is not symmetric, we might want to provide a point estimate for $\sigma$ using expectation values, instead of finding the most probable value. The integrals are nasty, but can be evaluated.
\begin{align}
    \langle \sigma \rangle = \int_0^\infty \mathrm{d}\sigma\,\sigma\,P(\sigma\mid D, I)
     = \frac{\Gamma\left(\frac{n-2}{2}\right)}{\Gamma\left(\frac{n-1}{2}\right)}\,\sqrt{\frac{n}{2}}\,r.
\end{align}
Alternatively, we could compute the expectation value for $\sigma^2$,
\begin{align}
    \langle \sigma^2 \rangle = \int_0^\infty \mathrm{d}\sigma\,\sigma^2\,P(\sigma\mid D, I)
     = \frac{n}{n-1}\,r^2 = \frac{1}{n-1}\sum_{i \in D}(x_i - \bar{x})^2,
 \end{align}
which may be familiar to you as the so-called sample variance, or the unbiased estimate of the variance. Really, by choosing to report the most probable value of $\sigma$, the $\langle \sigma \rangle$, or $\sqrt{\langle \sigma^2\rangle}$, we are just choosing one property of $P(\sigma\mid D, I)$ to report. We actually know the whole distribution, though, so whatever we choose is just a summary of it.

% %%%%%%%%%%%%%%%%%%%%%%%%%%%%%%%%%%%%%%%%%%%%%%%%%%%%%%%%%%%%%%%%%%%%%%%%

\pagebreak

% %%%%%%%%%%%%%%%%%%%%%%%%%%%%%%%%%%%%%%%%%%%%%%%%%%%%%%%%%%%%%%%%%%%%%%%
\section{The theory of Markov chain Monte Carlo}
\label{sec:mcmc}


% %%%%%%%%%%%%%%%%
\subsection{Why MCMC?}
When doing Bayesian analysis, our goal is very often to compute a
posterior distribution.  For most inference problems, the posterior
itself is the holy grail.  However, just having an analytical
expression for the posterior is of little use if we cannot understand
and properties about it.  Importantly, we often want to marginalized
the posterior; that is, we want to integrate over parameters we are
not interested in and get simpler distributions for those we are.
This is often necessary to understand all but the simplest models.
Doing these marginalizations requires what David MacKay calls ``macho
integration,'' which is often impossible to do analytically.

MCMC allows use to \textbf{sample} out of an arbitrary probability
distribution, which includes pretty much any posterior we could write
down.\footnote{Well, not \textit{any}.  For some cases, we may not be
  able to make a transition kernel that satisfies the necessary
  properties, which I describe in the following pages.}  We can
trivially perform marginalizations from these samples and can generate
histograms to plot various marginalizations.  All we have to do is
specify the distribution we want to sample from, and a good MCMC
algorithm will take care of the rest.

Generating samples that actually come from the probability
distribution of interest is not a trivial matter.  We will discuss how
this is accomplished through MCMC.


% %%%%%%%%%%%%%%%%
\subsection{The basic idea behind MCMC}
We often draw \textit{independent} samples from a \textbf{target
  distribution}.  For example, we could use
\verb|np.random.uniform(0, 1, 100)| to draw 100 independent samples
from a uniform distribution on the domain $[0,1]$.  Generating
independent samples for complicated target distributions is difficult.

But, the samples need not be independent!  Instead, we only need that
the samples be generated from a process that generates samples from
the target distribution in the correct proportions.  In the case of
the parameter estimation problem, this distribution is the posterior
distribution for model $A$ parametrized by $\mathbf{a}$,
$P(\mathbf{a}\mid D, A, I)$.  For notational simplicity, since we know
we are always talking about a posterior distribution, we will use
$P(\mathbf{a})$ for shorthand notation.

The approach of MCMC is to take random walks in parameter space such
that the probability that a walker arrives at point $\mathbf{a}$ is
proportional to $P(\mathbf{a})$.  This is the main concept and is
important enough to repeat.

\framebox[0.95\linewidth]{
  \begin{minipage}{0.9\linewidth}
The approach of MCMC is to take random walks in parameter
  space such that the probability that a walker arrives at point
  $\mathbf{a}$ is proportional to $P(\mathbf{a})$.\end{minipage}}

If we can achieve such a walk, we can just take the walker positions
as samples from the distributions.  To implement this random walk, we
define a \textbf{transition kernel},
$T(\mathbf{a}_{i+1}\mid \mathbf{a}_i)$, the probability of a walker
stepping from position $\mathbf{a}_i$ in parameter space to position
$\mathbf{a}_{i+1}$.  The transition kernel defines a \textbf{Markov
  chain}, which you can think of as a random walker whose next step
depends only on where the walker is right now; i.e., it has no memory.

The condition that the probability of arrival at
point $\mathbf{a}_{i+1}$ is proportional to $P(\mathbf{a}_{i+1})$ may
be stated as
\begin{align}
P(\mathbf{a}_{i+1}) = \int \mathrm{d}\mathbf{a}_i\, T(\mathbf{a}_{i+1}\mid \mathbf{a}_i)\,
P(\mathbf{a}_i).
\label{eq:invariance}
\end{align}
Here, we have taken $\mathbf{a}$ to be continuous. Were it discete, we just replace the integral with a sum.
When this relation holds, it is said that the target distribution is
an \textbf{invariant distribution} or \textbf{stationary distribution}
of the transition kernel.  When this invariant distribution is unique,
it is called a \textbf{limiting distribution}.  We want to choose our
transition kernel $T(\mathbf{a}_{i+1}\mid \mathbf{a}_i)$ such that
$P(\mathbf{a})$ is limiting.  This is the case if equation
\eqref{eq:invariance} holds and the chain is \textbf{ergodic}.  An
ergodic Markov chain has the following properties:
\begin{enumerate}
\item It is \textbf{aperiodic}.  A periodic Markov chain can only
  return to a given point in parameter space after $k$, $2k$,
  $3k,\ldots$ steps, where $k$ is the period.  An aperiodic chain is
  not periodic.
\item It is \textbf{irreducible}, which means that any point in
  parameter space is accessible to the walker from any other.
\item It is \textbf{positive recurrent}, which means that the walker
  will surely come revisit any point in parameter space in a finite
  number of steps.
\end{enumerate}

So, if our transition kernel satisfies this checklist and equation
\eqref{eq:invariance}, it will eventually sample the posterior
distribution.  We will discuss how to come up with such a transition
kernel in a moment, for for now we focus on the important concept of
``eventually'' in the preceding sentence.

% %%%%%%%%%%%%%%%%
\subsection{Burn-in}
Imagine for a moment that we devised a transition kernel that
satisfies the above properties.  Say we start a walker at position
$\mathbf{a}_0$ in parameter space and it starts walking according to
the transition kernel.  It may not reach a place in parameter space
where it is sampling from the limiting distribution.  This is because
the invariance condition, equation \eqref{eq:invariance}, does not
hold for every set of parameter values.  Once the walker reaches the
limiting distribution, it is indeed sampling from it.  So, we need to
let the walker walk for a while without keeping track of the samples
so that it can arrive at the limiting distribution.  This is called
\textbf{burn-in}, though Andrew Gelman and coauthors (in their famous book, \textit{Bayesian Data Analysis}) prefer to call it ``warm up.''

There is no general way to tell if a walker has reached the limiting
distribution, so we do not know how many burn-in steps are necessary.
There are several heuristics.  For example, Gelman and
coauthors proposed generating several burn-in chains and computing the $R$
statistic,
\begin{align}
R = \frac{\text{variance between the chains}}{\text{mean variance within the chains}}.
\end{align}
Limiting chains have $R \approx 1$, so you can use this as a metric
for having achieved stationarity.


% %%%%%%%%%%%%%%%%
\subsection{Generating a transition kernel: The Metropolis-Hastings algorithm}
The \textbf{Metropolis-Hastings algorithm} covers a widely used class
of algorithms for MCMC sampling.  I will first state the algorithm
here, and then we will show that it satisfies the necessary conditions
for the walkers to be sampling out of the target posterior
distribution.

\subsubsection{The algorithm/kernel}
Say our walker is at position $\mathbf{a}_i$ in parameter space.
\begin{enumerate}
\item We randomly choose a candidate position $\mathbf{a}'$ to step
  next from an arbitrary \textbf{proposal distribution}
  $K(\mathbf{a}'\mid \mathbf{a}_i)$.
\item We compute the \textbf{Metropolis ratio},
  \begin{align}
    r = \frac{P(\mathbf{a}')\,K(\mathbf{a}_i\mid \mathbf{a}')}
    {P(\mathbf{a}_i)\,K(\mathbf{a}'\mid \mathbf{a}_i)}.
  \end{align}
\item If $r \ge 1$, accept the step and set
  $\mathbf{a}_{i+1} = \mathbf{a}'$.  Otherwise, accept the step with
  probability $r$.  If we do reject the step, set
  $\mathbf{a}_{i+1} = \mathbf{a}_i$.
\end{enumerate}
The last two steps are used to define the transition kernel
$T(\mathbf{a}_{i+1}\mid \mathbf{a}_i)$.  We can define the acceptance
probability of the proposal step as
\begin{align}
\alpha(\mathbf{a}_{i+1}\mid \mathbf{a}_i) = \min(1, r) = \min\left(1, \frac{P(\mathbf{a}_{i+1})\,K(\mathbf{a}_i\mid \mathbf{a}_{i+1})}
    {P(\mathbf{a}_i)\,K(\mathbf{a}_{i+1}\mid \mathbf{a}_i)}\right).
\end{align}
Then, the transition kernel is
\begin{align}
T(\mathbf{a}_{i+1}\mid \mathbf{a}_i) = \alpha(\mathbf{a}_{i+1}\mid \mathbf{a}_i)\,K(\mathbf{a}_{i+1}\mid \mathbf{a}_i).
\end{align}


\subsubsection{Detailed balance}
This algorithm seems kind of nuts!  How on earth does this work?  To
investigate this, we consider the joint probability,
$P(\mathbf{a}_{i+1}, \mathbf{a}_i)$, that the walker is at
$\mathbf{a}_i$ and $\mathbf{a}_{i+1}$ at sequential steps.  We can
write this in terms of the transition kernel,
\begin{align}
P(\mathbf{a}_{i+1}, \mathbf{a}_i) &= P(\mathbf{a}_i)\,T(\mathbf{a}_{i+1}\mid \mathbf{a}_i) \nonumber \\
 &= P(\mathbf{a}_i)\,\alpha(\mathbf{a}_{i+1}\mid \mathbf{a})\,K(\mathbf{a}_{i+1}\mid \mathbf{a}_i) \nonumber \\
&= P(\mathbf{a}_i)\,K(\mathbf{a}_{i+1}\mid \mathbf{a})\,\min\left(1, \frac{P(\mathbf{a}_{i+1})\,K(\mathbf{a}_i\mid \mathbf{a}_{i+1})}
    {P(\mathbf{a}_i)\,K(\mathbf{a}_{i+1}\mid \mathbf{a}_i)}\right) \nonumber \\
&= \min\left[P(\mathbf{a}_i)\,K(\mathbf{a}_{i+1}\mid \mathbf{a}_i),
P(\mathbf{a}_{i+1})\,K(\mathbf{a}_i\mid \mathbf{a}_{i+1})\right] \nonumber \\
&= P(\mathbf{a}_{i+1})\,K(\mathbf{a}_i\mid \mathbf{a}_{i+1})\,\min\left(1,\frac{P(\mathbf{a}_i)\,K(\mathbf{a}_{i+1}\mid \mathbf{a}_i)}{P(\mathbf{a}_{i+1})\,K(\mathbf{a}_i\mid \mathbf{a}_{i+1})}\right) \nonumber \\
&= P(\mathbf{a}_{i+1})\, \alpha(\mathbf{a}_i\mid \mathbf{a}_{i+1})\,K(\mathbf{a}_i\mid \mathbf{a}_{i+1})\, \nonumber \\
&= P(\mathbf{a}_{i+1})\, T(\mathbf{a}_i\mid \mathbf{a}_{i+1}).
\end{align}
Thus, we have
\begin{align}
P(\mathbf{a}_i)\,T(\mathbf{a}_{i+1}\mid \mathbf{a}_i) = P(\mathbf{a}_{i+1})\, T(\mathbf{a}_i\mid \mathbf{a}_{i+1}).
\end{align}
This says that the rate of transition from $\mathbf{a}_i$ to
$\mathbf{a}_{i+1}$ is equal to the rate of transition from
$\mathbf{a}_{i+1}$ to $\mathbf{a}_i$.  In this case, the transition
kernel is said to satisfy \textbf{detailed balance}.

Any transition kernel that satisfies detailed balance has
$P(\mathbf{a})$ as an invariant distribution.  This is easily shown.
\begin{align}
\int \mathrm{d}\mathbf{a}_i \,P(\mathbf{a}_i)\,T(\mathbf{a}_{i+1}\mid \mathbf{a}_i)
&= \int \mathrm{d}\mathbf{a}_i\,P(\mathbf{a}_{i+1})\, T(\mathbf{a}_i\mid \mathbf{a}_{i+1}) \nonumber \\
&= P(\mathbf{a}_{i+1})\left[\int \mathrm{d}\mathbf{a}_i\, T(\mathbf{a}_i\mid \mathbf{a}_{i+1})\right] \nonumber \\
&= P(\mathbf{a}_{i+1}),
\end{align}
since the bracketed term is unity because the transition kernel is a
probability.

Note that all transition kernels that satisfy detailed balance have an
invariant distribution.  (If the chain is ergodic, this is a limiting
distribution.)  But not all kernels that have an invariant
distribution satisfy detailed balance.  So, detailed balance is a
sufficient condition for a transition kernel having an invariance
distribution.


\subsubsection{Choosing the transition kernel}
There is an art to choosing the transition kernel.  The original
Metropolis algorithm (1953), took
$K(\mathbf{a}_{i+1}\mid \mathbf{a}_i) = 1$. As a rule of thumb, you want to choose a proposal distribution such that you get an acceptance rate of about 0.4. If you accept every step, the walker just wander around and it takes a while to get to the limiting distribution. If you reject too many steps, the walkers never move, and it again takes a long time to get to the limiting distribution. There are tricks to ``tune'' the walkers to achieve this target acceptance rates.

Gibbs sampling, which is
popular, though we will not go into the details, is a special case of
a Metropolis-Hastings sampler, as is the No U-turn sampler (NUTS).
These both result in significant performance improvements over
important subclasses of problems.  The sampler employed by
\texttt{emcee}, the affine invariant ensemble sampler (Goodman and
Weare, \textit{J. Comp. Sci.}, \textbf{5}, 65--80, 2000), utilizes
many walkers walking at the same time, sharing information between
them.  It is technically not a Metropolis-Hastings sampler, but many
of the ideas presented in this lecture there apply for ensuring that
the sampler is indeed sampling the appropriate posterior distribution.

Finally, importantly, the No U-Turn sampler and the affine invariant sample can only handle continuous variables; they cannot sample discrete variables. Depending on your problem, this could be a serious limitation.

% %%%%%%%%%%%%%%%%%%%%%%%%%%%%%%%%%%%%%%%%%%%%%%%%%%%%%%%%%%%%%%%%%%%%%%%%
%
% \pagebreak
%
% % %%%%%%%%%%%%%%%%%%%%%%%%%%%%%%%%%%%%%%%%%%%%%%%%%%%%%%%%%%%%%%%%%%%%%%%
% \section{Parellel tempering MCMC}
% \label{sec:parellel_tempering}
% In this lecture, we will discuss parallel tempering Markov chain Monte
Carlo (PTMCMC).  This technique allows for effective sampling of
multimodal distributions and it avoids getting trapped on local maxima
of the posterior.

% %%%%%%%%%%%%%%%%%
\subsection{The basic idea}
Recall that the posterior distribution we seek to sample in the model
selection problem is
\begin{align}
P(\mathbf{a}_i\mid D, M_i, I) \propto P(\mathbf{a}_i \mid M_i, I)
P(D\mid \mathbf{a}_i,M_i, I).
\end{align}
Now, we define
\begin{align}
\pi(\mathbf{a}_i\mid D, M_i, \beta, I) &= P(\mathbf{a}_i \mid M_i, I)
\left[P(D\mid \mathbf{a}_i,M_i, I)\right]^\beta \\
&= P(\mathbf{a}_i \mid M_i, I) \exp\left\{\beta \ln P(D\mid \mathbf{a}_i,M_i, I)\right\}.
\end{align}
Here, $\beta \in (0, 1]$ is an ``inverse temperature'' in analogy to
statistical mechanics, where $-\ln P(D\mid \mathbf{a}_i,M_i, I)$ is an
energy (so $P(D\mid \mathbf{a}_i,M_i, I)$ is analogous to a partition
function).

If $\beta$ is close to zero (the ``high temperature'' limit), we are
just sampling the prior.  If $\beta = 1$, we are sampling our target
posterior, the so-called ``cold distribution.''  So, lowering $\beta$
has the effect of flattening the posterior distribution.  Therefore,
walkers at higher temperature (lower $\beta$) are not trapped at local
maxima.  By occasionally swapping walkers from adjacent temperatures,
we can effectively sample a broader swath of parameter space.

In practice, we choose a set of $\beta$'s with
$\beta = \left\{\beta_0, \beta_1, \ldots, \beta_m\right\}$, with
$\beta_{i+1} < \beta_i$ and $\beta_0 = 1$.

propose a swap roughly every $n_s$ steps.  To do this, we choose a
uniform random number on $[0,1]$ every iteration and propose a step
when this random number is less than $1/n_s$.  When we do propose a
swap, we randomly pick a temperature $\beta_j$ from
$\{\beta_1, \beta_2, \ldots \beta_m\}$.  We then compute
\begin{align}
r = \min \left(1, 
\frac{\pi(\mathbf{a}_{i,j}\mid D, M_i, \beta_{j-1}, I)}{\pi(\mathbf{a}_{i,j-1}\mid D, M_i, \beta_{j-1}, I)}\,
\frac{\pi(\mathbf{a}_{i,j}\mid D, M_i, \beta_j, I)}{\pi(\mathbf{a}_{i,j-1}\mid D, M_i, \beta_j, I)}\right).
\end{align}
We draw another uniform random number on $[0,1]$ and accept the swap
is that number if less than $r$.

This useful technique is implement in \texttt{emcee.PTSampler}, which
we will use in the next tutorial.  Conveniently, it automatically
chooses reasonable values of $\beta$ and swapping rate, though you can
choose these as well.


% %%%%%%%%%%%%%%%%%
\subsection{Model selection with PTMCMC}
We will now do some clever ticks to see how we can use PTMCMC to do
model selection without making the approximations we did earlier.
Recall the statement of Bayes's theorem for the model selection
problem, equation \eqref{eq:model_selection_bayes}.
\begin{align}
P(M_i\mid D, I) = \frac{P(D\mid M_i, I)\,P(M_i\mid I)}{P(D\mid I)}.
\end{align}
The likelihood is given by the evidence from the parameter estimation
problem, as we derived in equation \eqref{eq:bayes_factor_integral},
to give
\begin{align}
P(M_i\mid D, I) = \frac{P(M_i\mid I)}{P(D\mid I)} \,
\left[\int \mathrm{d}\mathbf{a}_i\,P(D\mid \mathbf{a}_i, M_i, I)\,
P(\mathbf{a}_i\mid M_i, I)\right].
\label{eq:bayes_factor_integral_2}
\end{align}
Now, we define a partition function
\begin{align}
Z_i(\beta) = \int \mathrm{d}\mathbf{a}_i\,P(\mathbf{a}_i\mid M_i, I)
\left[P(D\mid \mathbf{a}_i, M_i, I)\right]^\beta.
\end{align}
Our goal is to compute $Z_i(1)$, since this is exactly the integral in
brackets in equation \eqref{eq:bayes_factor_integral_2}.

Now, we're going to do a usual trick in statistical mechanics: we will
differentiate the log of the partition function (analogous to the
derivative of a free energy).
\begin{align}
\frac{\partial}{\partial \beta} \,\ln Z_i(\beta) &=
\frac{1}{Z_i(\beta)}\,\frac{\partial Z_i}{\partial \beta} \nonumber \\
&= \frac{1}{Z_i(\beta)}\,\int \mathrm{d}\mathbf{a}_i\, 
\frac{\partial}{\partial\beta}\,
\exp\left\{\ln P(\mathbf{a}_i\mid M_i, I) + \beta \ln P(D\mid \mathbf{a}_i, M_i, I)\right\} \nonumber \\
&= \frac{1}{Z_i(\beta)}\,\int \mathrm{d}\mathbf{a}_i\, \ln P(D\mid \mathbf{a}_i, M_i, I)\,
\exp\left\{\ln P(\mathbf{a}_i\mid M_i, I) + \beta \ln P(D\mid \mathbf{a}_i, M_i, I)\right\} \nonumber \\
&= \frac{1}{Z_i(\beta)}\int\mathrm{d}\mathbf{a}_i\,\ln P(D\mid \mathbf{a}_i, M_i, I)\,P(\mathbf{a}_i\mid M_i, I)\left[P(D\mid \mathbf{a}_i, M_i, I)\right]^\beta \nonumber \\
&= \left\langle \ln P(D\mid \mathbf{a}_i, M_i, I)\right\rangle_\beta,
\end{align}
where the averaging is done over the distribution
$\pi(\mathbf{a}_i\mid D, M_i, \beta, I)$, and the subscript $\beta$
indicates that the averaging is done for a specific value of $\beta$.
We can integrate both sizes of this equation to give
\begin{align}
\int_0^1 \mathrm{d}\beta\, \frac{\partial}{\partial \beta} \,\ln Z_i(\beta)
= \ln Z_i(1) - \ln Z_i(0) = \int_0^1\mathrm{d}\beta\,\left\langle \ln P(D\mid \mathbf{a}_i, M_i, I)\right\rangle_\beta.
\end{align}
Now, if the prior is normalized, as it should be,
\begin{align}
Z_i(0) = \int \mathrm{d}\mathbf{a}_i\,P(\mathbf{a}_i\mid M_i, I) = 1,
\end{align}
which means $\ln Z_i(0) = 0$.  Thus, we get
\begin{align}
\ln Z_i(1) = 
\int \mathrm{d}\mathbf{a}_i\,P(D\mid \mathbf{a}_i, M_i, I)\,
P(\mathbf{a}_i\mid M_i, I)
=
\int_0^1\mathrm{d}\beta\,\left\langle \ln P(D\mid \mathbf{a}_i, M_i, I)\right\rangle_\beta.
\end{align}
Fortunately, we have done MCMC, so we can easily compute the integrand
for each $\beta$ from our samples.
\begin{align}
\left\langle \ln P(D\mid \mathbf{a}_i, M_i, I)\right\rangle_\beta
= \frac{1}{n_\text{samples}}\,\sum_\text{samples} \ln P(D\mid \mathbf{a}_i, M_i, \beta, I).
\label{eq:ptmcmc_average}
\end{align}
Since we had to compute the log likelihood for every step, we have all
we need.  We then simple perform numerical quadrature across the
values of $\beta$ that we sampled to get the integral.  We therefore
can compute the odds ratio of two models $M_i$ and $M_j$ as
\begin{align}
O_{ij} = \frac{P(M_i\mid I)}{P(M_j\mid I)}\,\frac{Z_i(1)}{Z_j(1)}
= \frac{P(M_i\mid I)}{P(M_j\mid I)}\,\exp\left\{\frac{\int_0^1\mathrm{d}\beta\,\left\langle \ln P(D\mid \mathbf{a}_i, M_i, I)\right\rangle_\beta}{\int_0^1\mathrm{d}\beta\,\left\langle \ln P(D\mid \mathbf{a}_j, M_j, I)\right\rangle_\beta}\right\},
\end{align}
where the last ratio is via numerical quadrature on results computed
directly from our PTMCMC traces using equation
\eqref{eq:ptmcmc_average}.  We can get $\ln Z_i(1)$ using the built-in\\
\verb|thermodynamic_integration_log_evidence()| method of an
\texttt{emcee.PTSampler}.

% % %%%%%%%%%%%%%%%%%%%%%%%%%%%%%%%%%%%%%%%%%%%%%%%%%%%%%%%%%%%%%%%%%%%%%%%%
%
% \pagebreak
%
% % %%%%%%%%%%%%%%%%%%%%%%%%%%%%%%%%%%%%%%%%%%%%%%%%%%%%%%%%%%%%%%%%%%%%%%%
% \section{Introduction to images}
% \label{sec:intro_to_images}
% This lecture was presented as a Jupyter notebook, available here:
% \url{http://bebi103.caltech.edu/2015/tutorials/l07_into_to_images.html}.
% % %%%%%%%%%%%%%%%%%%%%%%%%%%%%%%%%%%%%%%%%%%%%%%%%%%%%%%%%%%%%%%%%%%%%%%%%
%
% \pagebreak
%
% % %%%%%%%%%%%%%%%%%%%%%%%%%%%%%%%%%%%%%%%%%%%%%%%%%%%%%%%%%%%%%%%%%%%%%%%
% \section{Hierarchical models}
% \label{sec:hierarchical_models}
% In this lecture, we will investigate \textbf{hierarchical models}, in
which some model parameters are dependent on others in specific ways.
This is best learned by example.

% %%%%%%%%%%%%%%%%%%
\subsection{A hierarchical model example}
In
\href{http://bebi103.caltech.edu/2015/tutorials/t3b_boolean_data.html}{Tutorial
  3b}, we studied reversals under exposure to blue light in
\textit{C. elegans} with Channelrhodopsin in two different neurons.
Let's consider one of the strains which contains a Channelrhodopsin in
the ASH sensory neuron.  We found that 9 out of 35 worms reversed
under exposure to blue light.  We used this measurement to estimate
the probability $p$ of reversal.  Specifically, we found that the
posterior probability of reversal given $r$ our of $n$ trials showed
reversals was\footnote{In Tutorial 3b, we used $n_r$ for the number of
  reversals.  We use $r$ here because we will have some more
  subscripts and we want to keep notation clean.}
\begin{align}
P(p\mid r, n, I) = \left\{\begin{array}{ccl}
\displaystyle{\frac{(n+1)!}{(n-r)!r!}\,p^{r}(1-p)^{n-r}} && 0\le p \le 1 \\[1em]
0 & & \text{otherwise}.
\end{array}
\right.
\label{eq:boolean_posterior}
\end{align}
This posterior assumed a uniform prior $P(p\mid I)$ on $0\le p \le 1$,
and a binomial likelihood, $P(r\mid n, p , I)$.

Next year, we will do the experiment again.  Actually, we could image
doing the experiment over and over again, each time getting a value of
$r$ and $n$.  Conditions may change from experiment to experiment.
For example, we may have different microscope set-ups, slight
differences in the strain of worms we're using, etc.  We are left with
some choices on how to model the data.


% %%%%%%%%%%%%%%%%%%
\subsubsection{Pooled data: identical parameters}
We could pool all of the data together.  In other words, let's say we
measure $r_1$ out of $n_1$ reversals in the first set of experiments,
$r_2$ out of $n_2$ reversals in the second set, etc., up to $k$ total
experiments.  We could pool all of the data together to get
\begin{align}
r = \sum_{i=1}^k r_i \text{ out of } n = \sum_{i=1}^k n_i \text{ reversals}.
\end{align}
We then compute our posterior as in equation
\eqref{eq:boolean_posterior}.  Here, the assumption is that the result
in each experiment are governed by \textit{identical parameters}.
That is to say that we assume $p_1 = p_2 = \cdots = p_k = p$.

This is similar to what we did in section \ref{sec:l01_learning}, in
which we looked at how a single hypothesis (or parameter value) is
informed by more data.


% %%%%%%%%%%%%%%%%%%
\subsubsection{Independent parameters}
As an alternative, we could instead say that the parameters in each
experiment are totally independent of each other.  In this case, we
assume that $p_1$, $p_2$, $\ldots$, $p_k$ are all independent of each
other.  Thus, the posterior probability is
\begin{align}
P(\mathbf{p}\mid \mathbf{r},\mathbf{n}, I) = \prod_{i=1}^k
\frac{(n_i+1)!}{(n_i-r_i)!r_i!}\,p_i^{r_i}(1-p_i)^{n_i-r_i},
\end{align}
where $\mathbf{p} = \{p_1, p_2, \ldots p_k\}$, with $\mathbf{n}$ and
$\mathbf{r}$ similarly defined, and the posterior is understood to be
zero if any the $p_i$'s fall our of the interval $[0,1]$.  

When we make this assumption, we often report a value of $p$ that is
given by the mean of the $p_i$'s with some error bar.


% %%%%%%%%%%%%%%%%%%
\subsubsection{Best of both worlds: a hierarchical model}
Each of these extremes have their advantages.  We are often trying to
estimate a parameter that is more universal than our experiments,
e.g., something that describes worms with Channelrhodopsin in the ASH
neuron generally.  So, pooling the experiments makes sense.  On the
other hand, we have reason to assume that there is going to be a
different value of $p$ in different experiments, as biological systems
are highly variable, not to mention measurement variations.  So, how
can we capture both of these effects.

We can consider a model in which there is a ``master'' reversal
probability, which we'll call $q$ to avoid too many $p$'s, and the
values of $p_i$ may vary from this $p$ according to some probability
distribution, $P(p_i\mid q, I)$.  So now, we have parameters
$p_1, p_2, \ldots, p_k$ and $q$.  So, the posterior can be written
using Bayes's theorem,
\begin{align}
P(q,\mathbf{p}\mid \mathbf{r}, \mathbf{n}, I) 
= \frac{P(\mathbf{r},\mathbf{n}\mid q, \mathbf{p}, I)\,
P(q, \mathbf{p}\mid I)}{P(\mathbf{n}, \mathbf{r}\mid I)}.
\end{align}
Note, though, that the observed values of $r$ do not depend directly
on $q$, only on $\mathbf{p}$.  In other words, they only depend
indirectly on $q$.  So, we can write
$P(\mathbf{r},\mathbf{n}\mid q, \mathbf{p}, I) =
P(\mathbf{r},\mathbf{n}\mid \mathbf{p}, I)$.  Thus, we have
\begin{align}
P(q,\mathbf{p}\mid \mathbf{r}, \mathbf{n}, I) 
= \frac{P(\mathbf{r},\mathbf{n}\mid \mathbf{p}, I)\,
P(q, \mathbf{p}\mid I)}{P(\mathbf{n}, \mathbf{r}\mid I)}.
\end{align}
Next, we can rewrite the prior using the definition of conditional
probability.
\begin{align}
P(q,\mathbf{p}\mid I) = P(\mathbf{p}\mid q,I)\, P(q\mid I).
\end{align}
Substituting this back into our expression for the posterior, we have
\begin{align}
P(q,\mathbf{p}\mid \mathbf{r}, \mathbf{n}, I) 
= \frac{P(\mathbf{r},\mathbf{n}\mid \mathbf{p}, I)\,
P(\mathbf{p}\mid q,I)\, P(q\mid I)}{P(\mathbf{n}, \mathbf{r}\mid I)}.
\end{align}
Now, if we read off the numerator of this equation, we see a chain of
dependencies.  The experimental results $\mathbf{r}$ depend on
parameters $\mathbf{p}$.  Parameters $\mathbf{p}$ depend on
\textit{hyperparameter} $q$.  Hyperparameter $q$ then has some prior
distribution.  Any model that can be written as a chain of
dependencies like this is called a \textbf{hierarchical model}, and
the parameters that do not directly influence the data are called
\textbf{hyperparameters}.

So, the hierarchical model captures both the experiment-to-experiment
variability, as well as the master regulator of outcomes.  Note that
the product $P(\mathbf{p}\mid q,I)\, P(q\mid I)$ comprises the prior,
and it is therefore independent of the data.


% %%%%%%%%%%%%%%%%%%
\subsection{Exchangeability}
The conditional probability, $P(\mathbf{p}\mid q, I)$, can take any
reasonable form.  In the case where we have no reason to believe that
we can distinguish any one $p_i$ from another prior to the experiment,
then the label ``$i$'' applied to the experiment may be exchanged with
the label of any other experiment.  I.e.,
$P(p_1, p_2, \ldots, p_k \mid q, I)$ is invariant to permutations of
the indices.  Parameters behaving this way are said to be
\textbf{exchangeable}.  A common (simple) exchangeable distribution is
\begin{align}
P(\mathbf{p}\mid q, I) = \prod_{i=1}^k P(p_i\mid q, I),
\end{align}
which means that each of the parameters is an independent sample out
of a distribution $P(p_i\mid q)$, which we often take to be the same
for all $i$.  This is reasonable to do in the worm reversal example.


% %%%%%%%%%%%%%%%%%%
\subsection{Choice of the conditional distribution/prior}
We need to specify our prior, which for this hierarchical model means
that we have to specify the conditional distribution,
$P(p_i\mid q, I)$, as well as $P(q\mid I)$ For the latter, we will
take it to be uniform on $[0, 1]$.  For the conditional distribution,
we will assume it is beta-distributed, which is defined on the
interval $[0,1]$ and can be peaked.  The beta distribution can be
written as
\begin{align}
P(p\mid \alpha, \beta) = \frac{\Gamma(\alpha+\beta)}{\Gamma(\alpha)\Gamma(\beta)}\,
p^{\alpha-1}(1-p)^{\beta-1},
\end{align}
where it is parametrized by positive constants $\alpha$ and $\beta$.
If $\alpha$ and $\beta$ are both greater than unity, the distribution
is peaked, and the mode is
\begin{align}
  p^* \equiv \omega = \frac{\alpha - 1}{\alpha + \beta - 2}.
\end{align}
The ``concentration,'' $\kappa = \alpha + \beta$, of the distribution
describes its spread.  As $\kappa$ gets larger, the distribution
becomes tighter.  So, we might want to think of the conditional
distribution in terms of $\omega$ and $\kappa$.  We can convert back
to $\alpha$ and $\beta$ using
\begin{align}
  \alpha &= \omega(\kappa - 2) + 1 \\
  \beta &= (1-\omega)(\kappa - 2) + 1.
\end{align}
We have $0 < \omega < 1$ and $2 < \kappa$.  A reasonable model would
be to take $\omega = q$ with some concentration $\kappa$.  This gives
an additional hyperparameter, $\kappa$, which describes
experiment-to-experiment variability.  We will take
$P(\kappa\mid I) \propto 1/\kappa$, as we typically do for scale
parameters.  Thus, our full posterior is
\begin{align}
P(q, \kappa,\mathbf{p}\mid \mathbf{r}, \mathbf{n},I) \propto
P(\mathbf{r},\mathbf{n}\mid \mathbf{p}, I)\,
\kappa^{-1}\,\left(\prod_{i=1}^k P(p_i\mid q,\kappa)\right),
\end{align}
nonzero on $0\le q,\mathbf{p} \le 1$ and $\kappa > 2$, where
\begin{align}
P(p_i\mid q, \kappa) =  \frac{\Gamma(\kappa)}{\Gamma(q(\kappa - 2) + 1)\Gamma((1-q)(\kappa - 2) + 1)}\,
p^{q(\kappa - 2)}(1-p)^{(1-q)(\kappa - 2)}.
\end{align}
As before, we have a binomial likelihood, where we assume the
experiments are independent.
\begin{align}
P(\mathbf{r},\mathbf{n}\mid \mathbf{p}, I) = \prod_{i=1}^k \frac{n_i!}{(n_i-r_i)!r_i!}\,
p_i^{r_i}(1-p_i)^{n_i-r_i}.
\end{align}


% %%%%%%%%%%%%%%%%%%
\subsection{Implementation}
In some cases, we can do some macho integration and work out
analytical results for the posterior of a hierarchical model.  This
usually involves choosing conjugate priors.  Most often, though, we
need to resort to numerical methods.  To see the worm reversal problem
solved with a hierarchical model, see the implementation
\href{http://bebi103.caltech.edu/2015/tutorials/l08_hierarchical_models.html}{here}.


% %%%%%%%%%%%%%%%%%%
\subsection{Generalization}
The worm reversal problem is easily generalized.  You can imagine
having more levels of the hierarchy.  This is just more steps in the
chain of dependencies that are factored in the prior.  For general
parameters $\boldsymbol{\theta}$ and hyperparameters
$\boldsymbol{\phi}$, we have
\begin{align}
P(\boldsymbol{\theta}, \boldsymbol{\phi} \mid D, I) = \frac{P(D\mid \boldsymbol{\theta}, I)\, P(\boldsymbol{\theta} \mid \boldsymbol{\phi}, I)\,P(\boldsymbol{\phi}\mid I)}
{P(D\mid I)}
\end{align}
for a two-level hierarchical model.  As we have seen in the course,
the work is all in coming up with the models for the likelihood
$P(D\mid \boldsymbol{\theta}, I)$ and prior,
$P(\boldsymbol{\theta} \mid \boldsymbol{\phi},
I)\,P(\boldsymbol{\phi}\mid I)$.
For coming up with the conditional portion of the prior,
$P(\boldsymbol{\theta} \mid \boldsymbol{\phi}, I)$, we often assume a
Gaussian distribution because this often describes
experiment-to-experiment variability.  Bayes's theorem gives you the
posterior, and it is then ``just'' a matter of computing it of
sampling from it.

% % %%%%%%%%%%%%%%%%%%%%%%%%%%%%%%%%%%%%%%%%%%%%%%%%%%%%%%%%%%%%%%%%%%%%%%%%



\end{document}
